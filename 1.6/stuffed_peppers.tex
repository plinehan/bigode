\mychapter
{chapter_carrots.jpg}
{stuffed peppers . couscous . roasted tomato salsa}
{sweet potato dill happening stuffed\\
\textit{into}\\
demure green peppers\\
\textit{over}\\
fragrant dry-fruit couscous\\
\textit{graced by}\\
roasted orange tomato salsa}

\section{stuffed green peppers with sweet potato dill filling}

there are only so many curries you can do before the other beachfront hotels 
(new york apartments?) complain about the smell (goddamn immigrants taking 
over this county!) and it's no longer sexy to eat with your hands anyway.

though it's still more civilized. contrary to the id\'{e}es 
re\c{c}ues\footnote{id\'{e}es re\c{c}ues (fr) : received ideas} of 
the famously civilized occident, eating with your right hand and (but not 
while) wiping with your left necessarily presuppose a higher baseline 
standard of hygiene --- your hands must be clean at all times in case eating 
or gnitae need to happen. from my brief sojourns There i got the sense that 
the average indian washed his hands almost fifty times a day.

carnaval week and all the country's transport had been appropriated for beer, 
so all i had for the crowds of brasilian vegetarians pounding on the blue 
metal doors fully three hours before the official meio dia\footnote{meio dia 
(br) : noon} opening were sweet potatoes --- perhaps leftover from biriyani 
(see chapter~\ref{chap:feijao}) or barbecued out back by the 
goddess murals --- and large green peppers, not too twisted and asking to be 
filled.

\begin{ingredients}
  \item one green bell pepper per human
  \item half as many sweet potatoes as humans (you'll have - leftovers, again)
  \item an onion for every three people
  \item ginger
  \item dill
  \item white wine if you're fancy, hedonistic, and not cooking for muslims
  \item suco de lim\~{a}o\footnote{suco de lim\~{a}o (br) : juice of lime}
\end{ingredients}

\subsubsection{making the filling}

chop your onions medium and ginger small. saut\'{e} them together (in butter, 
why not, live a little, it's ultimately healthier to realize your temptations 
than to suppress them) on low, slow heat. instead of frying brown, they will 
cook slowly and sweeten, browning only after half an hour to show their true 
sugary nature. if you don't have a half-hour to burn, i would recommend 
reorganizing your life to demonstrate your commitment to personal fulfillment 
and the primary importance of food to your physical and emotional well-being.

short of that, when you get impatient, turn up the heat and add a little 
liquid (milk, water, wine) to steam out the rest of the rawness. when the 
liquid is gone, the onions will be ready.

add the dill first, cooking down and mixing well. when you see a good painting 
of green on white (my operational hypothesis: your tastebuds and eyes are so 
unified in their aesthetic sensibilities that the right color balance will 
perfectly indicate the optimal dill strength) add the sweet potatoes --- they 
should already be boiled, steamed, baked, barbequed or otherwise wrangled from 
their natural state by the hands of men --- to the onions and stir firmly, 
mixing the halves together (they were always destined to be one, 
\textit{selon} aristophanes) in warmth and brotherhood.

you want the potatoes in small enough chunks to fit comfortably within a 
pepper, but not so pur\'{e}ed it looks recycled. it's going to be a little on 
the dry side and you can add lemon juice (not more than one lemon's worth) and 
any leftover white wine (according to the proclivities and majority status of 
your dinner guests) to the filling.

if it's still cloudy or taboo don't go another minute thinking that you can't 
or shouldn't add white wine or whatever alcohol you want really to any food at 
any stage in the cooking process. you can and should, subject to personal 
morality of course.

unless it's for a salad dressing most of the alcohol will cook off anyhow, but 
alcohol is particularly good

\begin{itemize}
  \item as a soup stock 
  \item in cooking risotto or rice in general (try \onehalf cup of sake in your coconut rice)
  \item to marinate cabbage
  \item in pasta sauce (red wine works well here)
  \item for steaming vegetables
  \item for slowly cooking onions, leeks, shallots, or garlic
\end{itemize}

if your daddy's not palling around with H. McTyiere Smith III, you'll probably 
want to use cheap leftover white wine that's going to vinegar anyhow, and even 
then you'll seldom need more than a cup. if the wine is really hecho 
mierda\footnote{hecho mierda (es) : turned to shit} just treat it like vinegar 
or even mix it in with one of your many flavored vinegar bottles.

\subsubsection{making the peppers}

as is generally the case, there is a traditional method used by indigen(i/o)us 
cultures with a deep appreciation of food complexity and lots of spare female 
labor time. also, you can boil them. if you choose the latter, just do so for 
a few minutes until they soften enough to be a little flexible. if you don't, 
you'll need live fire or a gas grill.

hold each pepper over the fire with tongs, chopsticks, or a couple of 
well-positioned forks. turn the pepper so the skin blackens evenly all around 
it. by the time the pepper is completely black it will be malleable enough to 
accept filling. now you just have to peel them.

\subsubsection{let the two become one}

with half-cooked empty peppers and a bowl of winey sweet-potato filling at 
your disposal, you're ready to start filling. preheat the oven, grease a tray, 
and decide whether you want the peppers to be top-stuffed or side-slit. the 
former involves pulling out the stem and seeds (perhaps with the help of a 
circular incision) and stuffing the pepper from above. for the latter 
technique, draw a slit down the pepper's thigh, allowing extraction of the 
seeds (shake it) and easier filling. this is usually an easier option, as the 
peppers will most likely be baking in a supine position.

most decadents --- you've already used the butter and the wine, why balk at 
the label? --- shine at the idea of adding cheese to the top of the filling 
and maybe over some of the pepper as well. as the peppers bake, the cheese 
will melt and the beginning of its browning will let you know Everything is 
Ready.

\subsection{the stuffed vegetable \textit{pattern} of culinary interaction}

so,\\
\mbox{\hskip1cm} welcome.

welcome to the next level: stuffing. stuffing is an ancient and justifiably 
famous technique for either

\begin{itemize}
  \item[a)] developing innovative taste and texture combinations in the context 
  of a microclimatic vegetable oven

  \item[b)] hiding leftovers or otherwise under-desirables inside another 
  vegetable, where it magically becomes ``special''
\end{itemize}

as with all choices, the real answer is ``neither'' or ``both''. so, both.

like casseroles and lasagnas, which are beneath the scope of this book, it's a 
great way to recycle leftovers and keep the culinary excitement fresh. you can 
reheat virtually anything that was good yesterday, mix in something perky 
(finely diced onion, lemon juice, cilantro), and use as stuffing with a little 
parmesan cheese on top.

in brasil, besides peppers (see above) we often used tomatoes, which require 
careful excavation and no-baking. in addition to standard get-down-in-there 
stuffing, O Bigode --- being situated a mere twenty feet from the historic 
colonial seaport of Aratuba --- was no stranger to vegetative boats. the 
planet and i both like the `nautical style' vegetable stuffing technique 
because it eliminates the need for flatware altogether. with a zucchini- or 
potato-boat the hungry customer sails away in a self-sufficient dining vessel, 
especially if they're comfortable wiping sans serviette.

\subsubsection{preparation for hard-nosed tubers:}
slice them in half, the long way, and boil or steam them until tender but not 
mushy --- between ten and twenty minutes depending on your karma but in all 
cases less than if you were planning on mashing anything.

\subsubsection{preparation for long squash:}
slice them in half, the long way, tracing your fingers wistfully along any 
curves, and bake them until tender but not mushy. you want to be able to pick 
up the squash with The One hand and eat out of it with The Other.

if the flesh isn't fully done but it's getting too soft, just Pull Out anyhow 
and remember that the two most popular t-shirts in bahia were (no translation 
necessary):

\mbox{\hskip2cm}\textsf{NO STRESS}\\
\mbox{\hskip1cm}and\\
\mbox{\hskip2cm}\textsf{100\% NEGRO}

let the ministry be a blessing. which is to say, you can still scoop it out 
and finish cooking it later. and cooking is just indulging your lazy stomach 
enzymes anyhow. 

for each vegetable you've built boats. eviscerate them (save the viscera) and 
gently massage the inside walls with assorted aspects of the Good: a little 
oil, a garlic clove, some salt, etc. then you can fill whatever suits your 
fancy (see any vegetable, bean, or grain recipe in this book) and bake to 
finish cooking. serve with a vegetable mayonnaise or other salsa, or atop a 
bed of grain (if you didn't use one inside).

the tomatoes are particularly great stuffed with something starchy --- 
potato-pea curry or mujaddra\footnote{mujaddra (arabic) : amanda's favorite 
lebanese dish: rice, lentils, and lemons together in love} have both been 
great. i like the potatoes more with onions and saut\'{e}ed greens. everything 
seems to work well in the zucchini.

as with the peppers above, most humans are suckers for highly concentrated 
melted cow fat. and in the project of treating everybody with love, from yeast 
to the moon, there's sometimes some juggling. even if you're normally not part 
of a horrifically murderous system that wrenches breastmilk out of cows with 
cold metal tentacles, sometimes it's good to make a stranger (or amanda) smile.

\subsection{variations on stuffed peppers}

\subsubsection{nutty zucchini boats}

for this i proceed as noted above, mixing the following together for filling:

\begin{ingredients}
  \item feta cheese
  \item finely chopped nuts (walnuts or almonds, depending on climate and mood)
  \item brown rice (usually leftover)
  \item fried onions
  \item grated carrots or golden beets
  \item a little lemon juice or white wine
\end{ingredients}

you can saut\'{e} the beets/carrots in ginger for a few minutes to soften or 
have them raw if you get off on that sort of thing. 

a hint for proportions --- if you have a large amount of brown rice, start 
mixing all the other ingredients together first and then add the rice so, 
while comprising the majority of the dish, it doesn't smother the taste and 
presence of all the fancy extras. if you only have a small amount leftover, 
start with the rice and add the extras accordingly, so they don't dominate and 
clash with each other.

crush the pepper over the top and bake until the zucchinis are ready to go.

\subsubsection{baked guacamole}

a pretty simple guacamole variation when you have the oven on and ready to go, 
cut open your avocados carefully, preserving both halves intact.

empty the avocado flesh into a bowl and mix in:

\begin{ingredients}
  \item one clove of garlic for each avocado pair
  \item a little lemon juice
  \item salt
  \item a very little cayenne pepper
  \item some bread crumbs, smashed up tortilla chips, or anything else crunchy and nutritionless
  \item \onehalf of the cheese of you want to eat
\end{ingredients}

fill the avocados once again with the mixture --- the added ingredients should 
more than compensate for the pit's sudden lack, giving you heaping avocado 
experiences. top with the other half of the cheese you wanted to eat and broil 
until you get a nice fragrant cro\^{u}te on top.
	
\section{fragrant couscous with raisins and cashews}

couscous, not a grain of its own, is a pasta made from semolina wheat, and 
popular throughout the middle east and north africa. it's a very small pasta 
and cooks almost instantaneously. i suspect that even very beginning 
meditators could cook it with their subtle vibrations.

the more traditional method is to pour boiling water over a bowl of couscous 
until the water level hits the couscous level (about equal quantities) and 
then cover immediately with a frisbee. mix in two bits of salt and olive oil 
into the `scous before the water hits to keep it loose and evenly salted. 
after five minutes of steaming, you can remove the lid and fluff. it should be 
soft, light, and airy. most people add too much water and the `scous comes out 
very dense. remember, you are both of the fire and of the air. honor that.

you can add other good stuff to the couscous before the boiling water or after 
cooking has completed. dry fruits (like raisins) will slightly rehydrate and 
become plump bursts of sweet awareness (generally a good thing). i like to add 
raisins and cashews, dried apricots and slivered almonds, or a small amount of 
ground turmeric and whole cloves. 

another diamond technique is to use a perfumed oil (rosemary and clove come to 
mind) or saut\'{e} some spices in olive oil on low heat for a few minutes (as 
the water boils perhaps) and use that olive oil to mix in the `scous before 
adding the water. i would do one stick of cinnamon and two cloves in this 
manner, and remove all plan(e)tary evidence before adding the oil.

\subsection{\textit{pattern} liner notes on cooking other grains}

most grains we cooked were of two styles --- rice and couscous. which is to 
say, they either require a lot of cooking (boiling, toasting, or baking) or 
just a little bit. when you're in a hurry or want to make a light meal, use 
the latter, which are generally smaller grains or products of larger grains.

of these, bulgur is the most popular and cooks just like couscous, except it 
takes 1.5 times its volume in water and wants fifteen minutes to cook.

when you have more time to cook and want a heartier grain, use whole wheat 
(not flour, the actual grains), barley, millet, or whatever else came out of 
the thresher. follow the instructions from the seller in terms of the water 
ratio and how much time to cook, but it's generally easy to check every ten 
minutes and see if your new rare and lovely grain is yet tender.

remember that it's all up to you, and you can use these grains in any of the 
above patterns, as well as develop your own ideas totally from scratch. from 
good ingredients and careful attention (a symptom of love), you can do no 
wrong. use bulgur as the object of your curries; use millet flour for half of 
your bread recipe; do fried rice with couscous. do it all. don't stop just 
keep going and know that every satisfied smile/belch combination is a small 
and pungent act towards a peace for the world's stomachs.

\section{roasted orange tomato salsa}

a lot of cooking is about color. this is a beautiful roasted tomato salsa that 
came out so nicely because the tomatoes were meio verde, meio 
madura\footnote{meio verde, medio madura (br) : not quite ripe}, a vibrant 
orange that might mean ``don't buy/use me yet'' if had we other options. we 
had not.

roast them quickly by heating olive oil on medium heat in a wide pan while you 
chop the tomatoes in half. they all go in one layer nestled together until you 
can smell the black bottoms (maybe between 5 and 10 minutes). you can use 
fingers, tweezers or tongs to give them a flip and maybe a tad more olive oil, 
shake the pan and let them grill and blacken in a few minutes.

if you don't like the black bitter papery edges of the skins, you can take the 
blisters off. i think they have a nice color, taste, and texture, so i'll chop 
up the tomatoes skin and all. this is a good salsa for the blender but in 
doing so you denature the intangible quality hand-dicing affords. a 
22\ordinal{nd} century comprise is to blend \twothirds of the tomatoes and 
carefully chop the remaining third (for texture, consistency, meditation) to 
add after blending.

dice the following to add to your roasted tomatoes, after blending:

\begin{ingredients}
  \item some onions (\onehalf to \onethird as much as your tomatoes)
  \item a few cloves/head of garlic (to taste, roasted tomatoes hold their garlic well)
  \item ground roasted cumin
  \item \onehalf bunch of chopped cilantro
  \item suco de 1-2 lim\~{a}o (depending on how sour those tomatoes were...)
\end{ingredients}

salt to taste.


\subsection{the roasted tomato salsa \textit{pattern}}

in the way that every man has his double and every puppet its shadow, every 
fresh salsa has its roasted analog. as most fresh salsas are based on tomatoes 
and chiles (of various species and colors), so too are most roasted ones. 
roasting gives a deeper, earthier flavor and can intensify the seasoning 
experience as well.

generally, when deciding to move a recipe from the fresh salsa pattern to the 
roasted one, the options emerge from deciding which elements will remain raw 
and which will be roasted. some roasted salsa are even reduced over the stove, 
unifying the flavors and cooking all of the constituents.

as always, experiment. anything you've done with fresh tomatoes or fresh 
chiles will be great with roasted tomatoes or roasted chiles. anything that 
comes from good ingredients will taste good in almost any form. we ourselves 
are the best instrument of our goals --- if you seek love you must act with 
love, and through love; if we seek peace, we must create peace and be peace.

\subsubsection{roasting chiles:}
is easiest in a heavy pan on the stove. without oil place your chiles on the 
hot surface, turning when the chile blackens. many slice open the chile and 
vacate the seeds after roasting, but i like to leave them all in for sake the 
of father coffman and the rest of the world's sufi voyagers.

\subsubsection{roasting sweet peppers:}
is best in the oven. oil them well and slice in half. place on a baking sheet 
close to the broiler under the highest heat. cook for ten minutes or until 
most of the skins are black --- the waiting is difficult, but every minute you 
wait will save you five-fold in peeling. when blackened, remove the tray and 
let cool, covered with plastic wrap: the steam will help loosen the skin. when 
the peppers have cooled, you can peel away the charred black, revealing a 
pulsing red belly underneath. this is the meat to be eaten naked(ly), dunked 
in oil, pur\'{e}ed, or in any other joyful form.

\subsubsection{roasting tomatoes:}
generally as above. the greener they are the longer you'll want to roast them 
and the more you'll want the lid --- so they steam more before total charring 
occurs.

\subsubsection{roasting eggplant:}
poke the eggplant in various places with a fork. cover with foil and bake for 
an hour (or until totally mushy) in the oven or, if you're near the flip side 
of a campfire, place the eggplant amongst the coals and leave it for a few 
campsongs. keep checking but the timing will vary greatly with the heat of the 
coals, but generally it will cook much quicker than in the oven.

let the roasted eggplant cool and cut in half. the pulp should easily scrape 
into whatever willing recipient you have. to get the true, smoky flavor that 
differentiates good roasted eggplant from bad, you must use a fireplace or buy 
large cans of ``mutabbel''\footnote{mutabbel (arabic) : eggplant} from the 
arab supply store.

\subsection{variations on the orange tomato salsa}

\subsubsection{roasted garlic tomato salsa}

\begin{ingredients}
  \item heads of garlic
  \item tomatoes
  \item whole cumin seeds
  \item lime juice and salt
\end{ingredients}

roast an entire head (or as many heads as you want) of garlic by cutting off 
the tips, massaging with olive oil, and wrapping in foil. place your package 
in the oven (best when something savory is already in there --- not 
recommended to accompany apple pie or cocadas).

after half an hour check to see if it's soft and gooey. it should be easy to 
peel and eager to be blended. chop your roasted tomatoes by hand, mix with the 
garlic and some ground roasted cumin. taste before adding salt and lime.

\subsubsection{roasted green tomato (tomatillo) salsa}

tomatillos are a mexican tomato variant that grow with adorable husks and have 
a unique flavor. in brasil we used unripe red tomatoes which were, in fact, 
green, and pretended they were tomatillos. the salsa tasted great.

\begin{ingredients}
  \item roasted green tomatoes (or tomatillos)
  \item roasted green chile
  \item raw or roasted garlic
  \item cilantro
  \item lemon juice and salt
\end{ingredients}

mix everything together in a mortar or blend in a few seconds with the robot. 
if you want it to be really fucking good, mix in half an avocado.

\subsubsection{mahamra\footnote{mahamra (arabic): the red one; red pepper 
spread}}

popular in lebanon where they need more sweet than spicy.

blend with post-war consumption levels of good green olive oil:

\begin{ingredients}
  \item a base of roasted red bell peppers
  \item a few roasted red chile peppers
  \item a raw or roasted garlic (c/g)love
  \item lemon juice
  \item salt
\end{ingredients}

\subsubsection{babaganoush\footnote{babaganoush (arabic) : roasted eggplant 
salsa}}

also a middle-eastern favorite. as indicated above, the real roasted eggplant 
flavor only comes from the extremes of culinary presence --- live fire or a 
lebanese can. your oven will not do the trick.

mix with a fork in a bowl large enough to handle some splashing:

\begin{ingredients}
  \item roasted eggplant innards
  \item tahini (a sizable but not dominant amount: keep tasting)
  \item finely chopped raw onion
  \item roasted garlic
  \item finely chopped tomato
  \item finely chopped parsley
  \item salt
  \item lemon juice (add last and make sure it's perfect)
\end{ingredients}

when it's just right, add a few teaspoons of good olive oil, mix in, and throw 
another teaspoon on top. garnish with paprika.
