\mychapter
{chapter_empadas.jpg}
{squash and potato empadas . pico de gallo . carrot hummus}
{butternut squash and madras potato empadas with fresh pico de gallo
and a bold carrot hummus}

\section{introduction to empadas}

the \gls{original economic thinking} behind O \tin{Bigode}
was to make and to give freely all the food with cumin in the air,
love in the heart, and no recommended price. the hope was to create an
autonomous space for post-capitalist economic relations, where humans
undertook commerce with gusto --- not points\index{points} or prestige --- as their
motivation.

barely a month after the grand opening, when the publicity machine was
still getting started, \tin{Bigode} collective members agreed to the first
AEA (Amandan\index{amanda} Economic Amendments) in an attempt to boost sales and to
adjust to local cultural norms. The AEA included:

\begin{itemize}
  \item adding fixed items to the menu with fixed prices, to be
  consumed at-the-bar or on-the-run

  \item expanding to include various juices in the lunch menu

  \item retaining the \gls{prato feito} at no fixed price

  \item revamping the accounting system for both the culinary and \tin{alcohol} divisions
\end{itemize}

the main result of the AEA was our decision to make two styles of
veggie empadas everyday, with one of the fillings tame and one of the
fillings exotic\index{exotic}, to reach out towards the comforts and sensibilities
of the island population. priced at merely a point and a half each,
they were cheap enough that houseguests wouldn't think twice about
buying half a dozen in a day, locals often got two or three, and even
children would scrape together the resources to share one.

empada is the portuguese translation of ``empanada'', a savory fried
or baked pastry popular all over latin america. we started each
restaurant morning by making the dough for the covers (usually three
dozen) and preparing the fillings while the dough rested. the oven
preheated (and usually roasted vegetables for that day's salsa or
vegetable main courses) while the bread rose and we filled the
empadas. generally we would have finished most of the cleaning by the
time the first round came out of the oven, a little before our noon
opening.

\subsubsection{how to make the covers:}

we started out using a recipe dona victoria (of epuyen, chubut,
argentina) had given me:

\begin{ingredients}
  \item 1 kilo de \gls{harina}
  \item 200 g de \gls{manteca}
  \item sal al gusto
  \item 600 mL de \useGlosentry{agua}{agua fria}
\end{ingredients}

and after a few rounds \tin{amanda} had it precisely at:

\begin{ingredients}
  \item \onehalf kilo cheap brasilian\index{brasil} white\index{color!white} flour
  \item \onehalf kilo expensive organic\index{organic} whole wheat flour
  \item 100 g of smelly unrefrigerated butter
  \item a seemingly equal volume of non-gmo soybean oil
\end{ingredients}

seeming! seeming? out on thee seeming, i shall write against it!

\begin{ingredients}
  \item enough water to make a smooth dough.
\end{ingredients}

you Knead the dough for a few minutes and let it rest for half an hour
until it comes into its truly elastic nature. in the waiting period,
you may perform whatever other duties call your attention in the
kitchen, garden, bar, or dining space, but please remember the ants
come fast for that strange smelly butter and your dough bowl had best
be isolated from the counter by at least a couple degrees of
separation.

i've weathered many arguments about ants in the kitchen, and
ultimately stand firm in my belief that our relations must be
co-operative. they communicate to us when we have cleaned well and
when we have been lazy, they volunteer to cart away any last chunks of
pineapple and cheese we missed, and they inspire us always to be
attentive. in the kitchen's cleanest days the only ants to be seen
were lazy scouts --- the morning after a heavy debauch saw more
wriggling bodies than counter space. the ants were a slow and
deliberate mirror of our own community, of ourselves.

when the dough sighs with contentment and the fillings are ready and
cooling, you may start the rolling. make sure everything\index{everything} --- the
surface, the rolling bin or wine bottle, your hands --- wears a
sportcoat of flour. check your rolling pin and surface for floury
bumps from last time --- they will cut your mambo like no other.

tear off a large jawbreaker of dough, flatten it into a disc with your
capable hands, and place on the board. roll into a perfect
circle. yes, yes, easier said than done. there should be some hints in
the chapatti section (see chapter four). with the empanadas it's
actually less important because a) no indian woman is going to laugh
at you b) bilateral symmetry comes into play more than radial
symmetry.

when you have a vaguely circular and strongly symmetric piece of
dough, transfer it off your rolling area to a specially allocated
filling plate. if it tears, it's too thin. place a heaping tablespoon
of filling in the middle and give it the eye. you need to be able to
fold one side of the cover over the filling and join it with the other
side. know that it will stretch a little in the folding and place
enough filling inside to be a competent snack. especially with
vegetarian food, nobody wants to pay for something that doesn't really
satisfy.

to join the two edges of dough together, rub a little water (it is
glue) on one side. if your dough is wet, you won't need this, of
course. then you have a number of options in terms of the
\gls{repulgada}. your first try
should be with the fork: like a pie crust, use the tines of the fork
to make consecutive impressions all around the joined halfmoon
perimeter.

later you can actually try braiding --- rather than look in a book or
the mystical internet, i suggest you find an old argentine (or other
latin american) woman and get her to show you. really, they're
everywhere, one will surely manifest herself if you're attentive to
the possibility. in a society whose primary social adhesive is
alienation, a little old-fashioned intergenerational cooking is in
itself an act of rebellion.

the empanada assembly process goes much faster with two people --- one
to roll and the other to fill-and-place. if Partner X is better she
can also take care of other kitchen events (flipping the oven, turning
down the pressure cooker, grinding the spices) while waiting for
Partner Y.

place your folded emp onto a greased baking tray. bake the full tray
for 10-15 minutes at medium (in brasil\index{brasil}, i soon learned, they have no
word for thermostat) heat, then flip all the emps and reinsert to
finish evenly. another 5-10 minutes (they start to brown when done)
until i-and-i will see them through.

alternately, you could deep fry them. yes, yes you could. another
variation: beat an egg in a bowl and brush a little egg onto each emp
before baking --- it will give the finished product a professional
sheen.

\section{empanada fillings: butternut squash and madras potato}

\subsubsection{\#1: exotic\index{exotic} butternut squash:}

\begin{ingredients}
  \item steamed or baked butternut squash
  \item an onion
  \item a few cloves
  \item ginger
  \item ground coriander
\end{ingredients}

follow the curry pattern, flavoring the oil with the cloves and saving
the chopped ginger (wet) and ground coriander (dry) for frying with
the onions. when the onions and spices have browned together, add the
chunks of squash (with a small amount of cooking water if available)
and mix together. after five minutes of cooking the flavors should be
well-acquainted: taste, salt, and taste again.

the filling, of course, should be moist but not runny.

\subsubsection{\#2: madras potatoes}

madras potatoes are made using a different currying pattern entirely
than the one presented earlier (see chapter four). no onions are used
and instead the soon-to-be-curried maiden is dropped directly into the
flavored oil.

\begin{algorithm}
  \item fry mustard seeds and dried red\index{color!red} chile in a little hot oil ---
  just enough to cover the pan.

  \item add 1 teaspoon or more of oil for each good-sized potato.

  \item as the new oil heats add in a scant teaspoon of cumin for each
  person eating, and \onehalf teaspoon of turmeric (total).

  \item when the cumin has browned, add your potatoes in bite-sized
  chunks. be sure the oil is hot or the potatoes will suck up all the
  oil. you will need to stir attentively or otherwise discipline them
  into not sticking to the pan and burning.

  \item continue to stir vigorously until the potatoes are cooked. you
  will have plenty of focused time to try stirring without a utensil
  (the flip method) or practice your portuguese. attend to other tasks
  at your peril.

  \item when the potatoes are done, salt, garnish with cilantro and
  serve (or set aside for empadas).
\end{algorithm}

\subsection{variations on making empanadas:}

almost every culture\index{culture} i've cooked in has a strongly rooted version of
the empada pattern. india is famous for its samosas (deep fried
tetrahedroids), el salvador for pupusas (stuffed corn tortillas), and
italy for calzones ( pizza folded in half). you can use the basic
rolling technique above for flatbread or yeasted doughs, cutting or
folding the circle to yield different shapes.

with respect to the inside of the equation, most of the rest of this
cookbook is about making the fillings. the possibilities, i have tried
to communicate, are literally endless. literally. not as in, ``oh, yes
mr. bigode\index{Bigode} there sure are a lot'' but there actually is no end, no
finitude, no opportunity to exhaust the different fillings that would
be awesome and appreciated in an empanada. it's nice to have two or
three on hand when making empanadas to appeal to different shades of
the personality whole. we did a different style of braid for each kind
so nobody gets confused.

rather than give recipes for each filling, i'll give a few and merely
indicate the pattern and ingredients involved for the rest. consider
it a test of sorts. empanadas involving corn, tomato, and potato were
most popular in brasil\index{brasil}, perhaps owing something to those vegetables'
nativity to the americas.

\subsubsection{tomato variations (generally with cheese):}
\begin{ingredients}
  \item raw with garlic, basil, black peeper
  \item saut\'{e}ed on low heat with italian seasonings (oregano,
  basil, rosemary)
  \item thrown in at the end of saut\'{e}ing onions with italian seasonings
\end{ingredients}

\subsubsection{corn variations (the corn was canned with some sugar added. don't tell anybody):}
\begin{ingredients}
  \item with garlic, spicy green chiles, and lime
  \item with garlic, salt, pepper, and cheese
  \item saut\'{e}ed with translucent onions and italian seasonings
  \item saut\'{e}ed with translucent onions and indian seasonings (cumin and coriander)
\end{ingredients}

\subsubsection{potato variations:}
\begin{ingredients}
  \item cutely diced potatoes and carrots with cumin and oregano
  \item cutely diced potatoes with italian seasonings
\end{ingredients}

\subsubsection{bean variations:}
\begin{ingredients}
  \item saut\'{e}ed greens (kale, spinach) and garlic mixed with
  drained leftover black beans
  \item adzuki beans refried with grated carrot and ginger
\end{ingredients}

\subsubsection{vegetable variations:}
\begin{ingredients}
  \item saut\'{e}ed grated carrot and cabbage with soy, garlic,
  ginger, and sesame
  \item xu-xu tropeiro
  \item fried plantains with (a little) sugar, cinnamon, and hot green
  chile
  \item curried eggplant with lots of cumin and parsley
  \item saut\'{e}ed onion and zucchini, reduced in beer
\end{ingredients}

\subsubsection{leftovers that might work well}
\begin{ingredients}
  \item refried black beans and cheese
  \item feij\~{a}o tropeiro
  \item roughly chopped fajita style vegetables
  \item drained ratatouille
  \item eggplant curry
\end{ingredients}

\section{\gls{pico de gallo}}

\begin{ingredients}
  \item 3:2 tomato to onion
  \item 1 bunch \gls{cebola verde}
  \item garlic
  \item green chile
  \item salt
  \item ground roasted cumin
  \item cilantro
  \item lime
  \item a touch of cheap-ass sugary brasilian\index{brasil} vinegar
\end{ingredients}

chop and combine. chop and combine. chop and combine. (there are
machines that do this --- see chapter zero)

\subsection{the fresh-cut salsa \textit{PATTERN}}

it's been generally recognized in the culinary world --- except for an
aberrant forty year period where microwaves were considered part of
the communal future --- that the best food comes from fresh
ingredients and is chopped quite small. naturally, with these
guidelines, you can't do better than a fresh, raw salsa.

what makes a good salsa really good is a synergy amongst its
ingredients. as in a well-designed dinner party, each participant
brings something different to the table --- some of whom complement
each other nicely, while others add unique and surprising elements.

in our poster child above, the tomato and onion form a polysensual
frame for the remainder of the ingredients. the tomato is soft, red,
and mild; the onion is crunchy, white\index{color!white}, and sharp. the cilantro and
green chile, both refreshing and green, serve to \tin{balance} the suave and
the pungent dimensions. the garlic adds gusto to onion's assault while
contributing its own unmistakable flavor. lime and salt ground the
salsa in their respective taste regions, giving the finished product a
\tin{balance} among salt, sour, and spicy.

finally, the fresh ground cumin adds a slightly sweet and slightly
bitter earthy depth to the food that penetrates deeper and holds
longer than the raw spices can manage. for the love of{love!for the love of} krishna's bevy
of virginal consorts, do not use old storebought ground cumin --- it
adds nothing but a stale dimension to your concoction.

to variate, think of replacing any of these elements with a similar
substitute, or eliminating/adding balanced\index{balance} ingredient
pairs. similarity usually falls into the following categories:

\begin{itemize}
  \item by appearance --- either vegetables which are the same color\index{color}
  (when prepared) as whatever you're replacing or vegetables which are
  equally Not the color\index{color} of whatever else is going in the dish

  \item by texture --- ingredients that look and taste totally
  different may still serve the same crunch reflex for your mouth.

  \item by size --- finely chopped or grated vegetables are easier to
  digest; a harder vegetable can replace a softer one if it is cut
  into smaller pieces.

  \item by taste --- naturally, if one element satisfies the same
  taste region as another, it is a perfect candidate for substitution.
\end{itemize}

often a substitution will make sense within the framework of multiple
categories. the key is to make sure you can justify the variation,
either through the external imposition of arbitrary rules, a personal
logic, or the courage of your whims. to taste excellent, a recipe (or
combination) need only be inspired, considered, or tested --- that is,
have the blessing of Holiness, Deduction, or Experience.

some common examples:

\begin{itemize}
  \item green or orange\index{color!orange} peppers can often be used instead of onion ---
  they share the color\index{color} of ``not-tomato'' and have crunchy texture. the
  flavor is of course totally different, usually on the sweet side,
  indicating one should add more garlic or chile to hold the
  pre-existing \tin{balance}.

  \item onions, green onions, and shallots taste similarly enough to
  be used in each other's recipes with proportional
  recalibration. similarly with lemon juice, lime juice, and vinegar.

  \item chipotle peppers, red\index{color!red} peppers, and tomatoes all share a red
  influence with widely different tastes and textures. their
  replacement signifies a major substitution (you will have a totally
  different salsa) but should always work well.

  \item fresh herbs vary totally in taste but offer a similar color\index{color},
  texture, and feeling of freshness to the dish. with the cilantro and
  parsley families, substitutions can be made with abandon, but
  branching out between basil, dill, mint, etc. generally require a
  deep recalibration of all the flavor attunements.

  \item one type of cooked bean (or corn) will usually have the same
  texture and acidity as another, leaving only color\index{color} (and taste if it
  concerns you) to be considered when switching.
\end{itemize}

\subsection{variations on pico de gallo}

\subsubsection{fresh chipotle salsa}

\begin{ingredients}
  \item 1 can canned chipotle peppers in canned adobo sauce
  \item two hefty onions
  \item half a head of garlic
  \item a few limes and some salt
  \item diced bell pepper or corn (optional)
\end{ingredients}

the chipotles are hot and this recipe calls for the whole thing just
so you don't end up wasting any. it's generally best to use half in a
strong chipotle salsa and use the rest with black beans or rice or in
a soup or something large and diffusive.

chop the chipotles into a pasty mess and mix in your salsa bowl with
evenly diced onions, minced or pounded garlic, the juice of a few
limes, and corn or corn-sized pieces of sweet bell pepper, if
using. salt to taste --- the corn or peppers are mainly for visual
stimulation and to help in the biomass effort to cut away the force of
the chipotles.

\subsubsection{beet apple salsa salad}

\begin{ingredients}
  \item as much beet as apple
  \item a small amount of finely chopped onion
  \item ground cloves
  \item soy sauce
  \item grated ginger
  \item apple cider vinegar
  \item honey
\end{ingredients}

in itself this recipe has four main versions, all of which taste
different and are appropriate in different contexts purely as a
function of chopping and texture. the main axis is of course the
\tin{balance} between the apple's light crispness and the dense, earthier
sweetness of the beets. either can be diced (brunoise) or grated and
mixing a grated One\index{One!and the Other} with a brunoise Other highlights the latter in a
fine bed of the former.

for use more as a salad, grate neither; if your vision heads towards a
dressing grate both. the border regions are the most interesting, of
course. when you've found your \tin{balance}, mix in a small amount of
dressing, based equally in soy sauce and vinegar and modulated with
honey in accordance with the sweetness of your produce. the ground
cloves, if fresh, should be used sparingly and operate subtly, while
the grated ginger heads the piquant direction and can be emphasized
accordingly.

\section{carrot hummus}

carrot hummus is a cross between a typical middleeastern hummus and a
carrot-based dressing you can pick up in new york or patagonia. it
displays typical ``hybrid vigor'' for most crowds, has more cream and
consistency than a straight-up carrot mayo, and more vibrancy and
psychological nutrition than the standard (and much loved) hummus.

as with any hybrid, you can look it at from the perspective of either
parent, generating two different recipes and tastes. i'll provide both
approaches.

from the east:

\begin{ingredients}
  \item cooked drained chick peas (with some water reserved)
  \item cooked carrots (with some water reserved)
  \item tahini
  \item lemon juice
  \item olive oil
  \item garlic (raw or roasted)
  \item toasted ground cumin seeds
  \item salt
  \item parsley garnish
\end{ingredients}

from the west:

\begin{ingredients}
  \item cooked carrots (with some water reserved)
  \item cooked drained chick peas (with some water reserved)
  \item rice wine vinegar
  \item sesame oil
  \item ginger
  \item toasted ground coriander
  \item salt
  \item cilantro garnish
\end{ingredients}

a good place to start is to use equal quantities (masses or sizes) of
the carrots and garbanzos beans. sometimes i'll serve this alongside a
typical hummus, in which case i'll use a much higher proportion of
carrots, with the garbanzos mainly for mass and creaminess. i've heard
in some of the remote provinces it's fashionable to do the reverse ---
make a ``flavored'' hummus that has just enough carrot for a bit of
color\index{color} and sweetness.

i use garbanzos i've soaked and pcooked\index{pcook} but of course the key thing is
that you're eating, so if they come in a can, ultimately it's all
sacred and nothing edible is to be disparaged. except the water from
the can, which can be spat upon and discarded. the carrots can be
cooked in any way that you like --- boiling and steaming are the
quickest, saut\'{e}ing slightly longer but better flavor, and roasting
the longest, but for me, the most rewarding. when i boil them i add
ginger to the water from the onset, and save the stock against an arid
day.

blend the two substances together. in a \tin{robot} this should be fine. in
a blender you will need some lube. this is a good time to add the sour
(lemon juice or vinegar) and tahini (if you're rolling in from the
east). if they still do not cream together, add some of the indicated
oil or stock water.

with respect to proportions, it's easy to add too much vinegar and
hard to add too much lemon. this is mainly because of the comparative
work involved in squeezing a lemon and squeezing a bottle of
vinegar. most recipes will call for \tin{XXXX} tablespoons of lemon juice
for \tin{XXXX} cups of hummus. for most people it's hard to add too much
oil, and if you're not one of those people, you would have gone with
the water.

add the wet and dry spices (ginger or garlic, cumin or coriander,
salt). these quantities are entirely to taste and there is much to be
gained in their combinatorial experimentation.

capsaicin addicts can work in their cayenne fix at any time. when
everything\index{everything} is mixed together, taste and see what should be
lacking. you should encounter a good \tin{balance} of acidity, piquancy,
earthiness, and cream. the creaminess --- provided by oil or tahini
--- is particularly important and DO NOT STOP ADJUSTING until you have
it right.

\subsection{\textit{patterns} of salsic combination}

thus far we've looked at a variety of salsa tropes: fruit salsa, chile
vinegars, bean dips, fat (coconut and peanut) sauces, pestos,
vegetable pur\'{e}es, roasted salsas, and the traditional fresh-cut
variety. after a few grinds in the \tin{mortar} and trips around the \tin{robot},
it becomes pretty obvious these techniques can and should be
mixed. indeed, people love the mix in flavor and texture that comes
with throwing saut\'{e}ed onions in a fresh salsa or roasted red
peppers in a bean dip.

it's often easier to start with a smoother salsa pattern --- the
vegetable mayo, roasted veggies, or hummus for example --- and work in
fresh and chunky elements. for example, added chopped tomato or onion
to a carrot mayonnaise brings you to a whole new level of fresh-salsa,
building off of a creamy carrot base instead of straight
tomatoes. inverting the notion of progress you could throw finely
diced carrots and sharp spring onions into a smooth roasted tomato
salsa for a similar textural effect.

another easier suite of variations is to mix two salsas together (the
day after perhaps) in even proportions. i've found this works
particularly well with mexican style salsa, fresh and roasted, based
on tomatoes, sweet peppers, and chile peppers. half roasted-tomatillo
(con \gls{palta}) and half
roasted-garlic chipotle makes a wicked empanada or enchilada sauce.

\subsection{variations on carrot hummus}

\subsubsection{slippery green onion salsa}

the difference between an easy-to-make salsa and a hard one, for me,
can be seen in the soiled dishes, as wind in the waves of a lake. this
salsa requires a lot of dishes. eventually you will mix the following
four elements together

\begin{ingredients}
  \item 2 broiled red\index{color!red} peppers
  \item 1 Tbsp ground toasted cumin seeds
  \item 1 bunch saut\'{e}ed green onions
  \item a mixture of garlic, ginger, lime, cilantro, and chile
\end{ingredients}

for the peppers:

these will take the longest and are the limiting step in terms of
speed. first, turn on the broiler. split them in half, excavate the
stem and seeds, and lightly oil. broil until the skins blister and
turn black. it will take ten minutes or more --- check back ritually
after you do each of the following tasks. when they are black, turn
off the oven, take them out and cover the tray with a plastic bag to
hold steam in and loosen the skins. refrigeration also works well. go
back to cooking. when everything\index{everything} else is done, they should be easy to
handle. discard the skins and blend or mash the flesh into a
pur\'{e}e. the smallest hint of vinegar can help.

for the cumin:

toast them without oil in a hot pan. when they darken, begin to smell
absolutely fucking wonderful, and start to stick, move them to a
\tin{mortar} and pestle (or coffee grinder) to pulverize. you may use the
same (hot) pan for the green onions.

for the green onions:

chop them finely and saut\'{e} over high heat in hot oil until they
just begin to brown. do not let them brown further, but cut their
mambo with some water. they will steam and hiss and come loose from
the pan. when the hissing stops, reduce the heat and cover, they will
steam to a tender slippery state.

for the raw power, dice or \tin{mortar} together

\begin{ingredients}
  \item 3 garlic cloves,
  \item \onehalf inch ginger,
  \item 1-2 fresh spicy chile
  \item the juice of 3 limes
\end{ingredients}

keep a bunch of cilantro washed and well chopped. if you don't use it
all now, you'll use it for something else. it's good for you.

combine the dry cumin, the onions, the pur\'{e}e, and the wet
spices. salt. mix in enough cilantro so it gives a strong presence but
does not overwhelm. the strong accents should be a power triangle of
dry cumin, juicy sweet peppers, and sharp piquancy, with the tamed
strength of the onions underlying it all.

\subsubsection{roasted tomato black eye pea salsa}

\begin{ingredients}
  \item olive oil and roast tomatoes
  \item blend them with black eye peas (2:1 tommy to pea)
  \item add in diced onions and garlic
  \item basil if you can get it
  \item fresh oregano or marjoram if you cant
  \item suco de lim\~{a}o
\end{ingredients}

tomatoes and most other fruits and vegetables are best raw. but if
you're taking this plunge and deciding to divest them of their
unadulterated beauty, roasting is the way to go. especially for
tomatoes and red\index{color!red} bell peppers.

i have two methods of roasting tomatoes, depending on what else is
going on in the kitchen.

\begin{description}
  \item[stovetop:] involves a heavy-bottomed pan on medium heat and
  tomatoes cut in half and rubbed with olive oil. when one side is
  black turn them over to cook the other side. this is easier in some
  way but steaming action happens as well and if you want the
  straight-up roasted flavor this is not for you. a plus is that you
  can empty the pan and it's hot and oily and ready to go for your
  eggplant or whatever else you have on the menu.

  \item[oven:] tomatoes stemmed and rubbed with olive oil on an oiled
  pan under the broiler. they will black and blister quickly so be
  FOCUSED, AWARE and ATTENTIVE, which is really what the kitchen dance
  is all about anyway. so it works out. flip tomatoes when black the
  first time and remove when black the second time.
\end{description}

in either case, let them cool before you peel them and then throw them
in a blender with some black-eyed peas. this recipe is perfect after
you've made acaraj\'{e} or refried black-eye peas or texas caviar and
have a bunch of cooked black eyes left over in the
\gls{geladeira}.

remove the blend and add in the onions and garlic you chopped while
the tomatoes were cooking. the onions should be diced small and the
garlic minced. salt to taste and finally add the chopped basil and
lime juice. lemon juice is probably even better but not so possible in
brasil\index{brasil} and really at some fundamental level of reality\index{reality}, past the
artifice of ``gourmet'', ``clash'', and ``palate'' it's all the same
damn thing anyway.

what we're doing with cooking --- i think --- is more than satisfying
hunger or even artistic expression. i think we're cultivating a new
type of human who will be the kernel of a new type of society. part of
that human must be aware that food is not only nourishing and
medicinal and good but also sacred, and he is a blessed being by
getting to eat at all. that's why i think it's important to eat what's
in season and to experiment playfully with limitations on your diet
--- the glory age of global gluttony might be fun (for some) while it
lasts but pretty soon we might have to eat root vegetables and
preserves all winter and it's going to be sad for the gourmands if
they don't learn to love it.

that said, i do include a gourmet version of this recipe. in my
experience, gourmet tends to indicate one of the following:

\begin{itemize}
  \item spectacular nomenclature for decidedly unspectacular reality\index{reality}
  \item involvement of a strange animal product
  \item inclusion of expensive or exotic\index{exotic} ingredient
  \item precise preparation required
\end{itemize}

since the first three options are philosophically unavailable to me,
this type of gourmet involves concentration and a sharp knife.

before roasting the tomatoes, slice them into thin circles. place
oiled cloves of garlic alongside them and watch them very
carefully. cut the broiled circles into quarters and mix with very
finely chopped onions. the garlic should be soft enough that chopping
slides into mashing --- mix them with the drained black-peas and lemon
juice (not lime, jeeves, but lemon) and a few drops of balsamic
vinegar (tell nobody and nobody will know) and add them to the
becoming-salsa. rinse the basil leaves and stack them. roll from the
tip end up to the base as if you were removing a carpet. then slice
the curled leaf into thin wispy segments (chiffonade) to toss in
lightly with the salsa. add the perfect amount of salt and serve with
some whole roasted tomato circles on top.
