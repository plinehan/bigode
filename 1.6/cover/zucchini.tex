\mychapter
{chapter_carrots.jpg}
{zucchini steaks . coconut-wheat bread . passionfruit hummus . tomato
salad} {broiled zucchini steak sandwiches on coconut-wheat bread,
served with passionfruit hummus and fresh vegetables while a
mediterranean-inspired tomato salad chills on the side}

\section{broiled zucchini steaks}

\begin{ingredients}
  \item zucchini
  \item olive oil
  \item lime juice
  \item hot chiles (optional)
  \item freshly ground black peppercorns
  \item salt
\end{ingredients}

vegetable steaks are simple, hearty, fast, and fulfilling. all you
need is a broiler.

move a rack to the top of the oven and preheat the broiler. trim the
ends of the zucchini (or whatever summer squash you might have) and
slice it lengthwise into steaks slightly smaller than the thickness of
your finger. the fatter they are the juicier and rawer they will be in
the middle. marinate them in a baking dish or tray with the rest of
the ingredients. in brasil\index{brasil}, lime juice is easier to find than lemon
--- the key is to find an acidic taste to \tin{balance} the oil. you can use
vinegar too: balsamic and red\index{color!red} wine both work great with squash.

clearly the longer they marinate the better (overnight would be
divine) but i'm assuming we're dealing with time pressures and bad
attitudes so it's generally good to start baking as soon as the oven
is hot enough. it's not a big timecommitment to do a few minor
preparations each night before going to bed --- soaking beans,
marinating vegetables, starting bread --- but somehow those rituals
have been excised from our cultural rituals and replaced by late night
television\index{television}. oh well.

place your zukes up close to the flame and broil them for a few
minutes. the first black spots indicate they're ready to go --- pay
careful attention and flip them as they get to the level of charring
you're comfortable with. the second side won't need as long, might be
rather wet, and sometimes if they're thin enough i don't even flip
them at all. move them to a plate where they can cool and await the
sandwich of their dreams.

\subsection{the lunch vegetable steak \textit{pattern} explicated}

the typical lunch/snack food in brasil\index{brasil}, served at lanchonetes every
twenty meters with tall glasses of freshly squeezed orange\index{color!orange} juice for
less than 1 point, is a variation on the ``X-Egg''. X naturally is
pronounced ``cheese'' and the ``cheese-egg'' is a fried egg and cheese
sandwich on old white\index{color!white} bread toasted with generous amounts of
butter. high quality establishments, like our neighborhood joint
``\useGlosentry{parade obrigatoria}{parade obrigat\'{o}ria}'' serve it up with lettuce, tomato, and picante.

or those of you planning on traveling to brasil\index{brasil} in this plane or the
astral one, i may note that the ``X-Egg'' is something of an emergent
property of language and cuisine, and neither the words ``Cheese'' nor
``Egg'' are necessarily intelligible on their own.

our complementary option was a vegetable sandwich with optional
``X-Egg'' on top of it. as it was a popular paradigm at O \tin{Bigode}, we
came up with a number of variations. the markets usually had either
eggplant, butternut squash, or zucchini, and the techniques would also
work well with root vegetables. the variations we indulged in were of
three types: choice of \tin{XXXX} vegetable, cooking technique, and
preparation.

the broiling zucchini number is fast and easy and works for most other
vegetables as well. the fleshy ones provide the best results of course
but slices of a large beet or rutabaga work great. starchy root
vegetables (potato, yam, manioc) in a sandwich get a little on the
chalky side for me and are best enjoyed on their own.

baking the vegetable, as is done in the squash recipe (below) takes a
little longer and gives a sweeter softer taste. it's also appropriate
if you want to throw a bunch of other flavors in with your main event:
the added time helps them break the ice; the lower heat keeps smaller
guys from burning as the larger ones finish cooking.

if you don't have an oven you can use a combination of dry roasting
and steaming to get a similar effect on the stove top. lightly grease
your pan or griddle, place a few steaks on medium heat, and cook until
it starts to smoke and blister. flip the steaks, cooking another
minute or two, and pour a small amount of water onto the
pan. immediately cover as the water steams and it should cook into
your steaks.

searing or breading the steak before baking it speeds up the cooking,
adds another dimension of flavor, and provides an interesting
texture. argentines and brasilians\index{brasil} alike are fond of dunking food in
egg and dredging it through flour before cooking, and the ubiquitous
mandioca used in brasil\index{brasil} adds a great crunch.

and yes, of course, eggs are neither considered vegetables nor
vegetarian.

\subsection{variations on broiled zucchini steaks}

\subsubsection{milanesa de berenjena with rosemary hummus and a gentle flower of baked garlic}

slice your eggplant lengthwise as you did the squash, thinner than a
finger in depth. many prefer the eggplant peeled but i'll leave it on
if it's organic\index{organic}. nutrients and pesticides alike congregate in the
\gls{piel}, which often has a totally different
(and usually more bitter) taste than the vegetable inside. if you
don't believe me try eating just the peel of a grape and see what you
think.

heat a pan on medium-high heat and prepare it with a little oil. if
you happen to have any rosemary-infused oil lying around, use
that. loosely beat together a couple of eggs (and maybe some water to
thin) in a shallow plate and assemble the following in another one:

\begin{ingredients}
  \item 1 cup \gls{farinha de mandioca} (or perhaps flour or cornmeal or any grainy grain product)
  \item 1 tsp salt
  \item 1 tsp freshly ground toasted cumin (or whatever cumin you have)
  \item \onequarter tsp cayenne pepper (if you're up for it)
\end{ingredients}

now take each steak, wet both sides in the egg-mixture, drip it over
to the flour plate, dredge it, and toss it into your pan or grill. it
will take a few minutes to cook each side --- the batter will stick to
the eggplant and brown --- so the bigger the pan the faster this task
will be.

as each steak finishes, move it to a greased baking sheet and bake as
you would the zucchini above or squash below. while you're waiting for
the eggplant to cook, take

\begin{algorithm}
  \item an entire head of garlic and
  \item cut off the tops (where they join together) and
  \item rub with olive oil and
  \item wrap in foil.
\end{algorithm}

now put that present in the preheating oven while you're waiting
for the steaks to finish. time it so that as the last eggplant
finishes the oven is the right temperature for baking the tray. at
this point the garlic should be half-way through its baking therapy
and you can pull it out in a few minutes when the milanesas are cured.

as with the zucchini, you may need to fold or cut an individual steak
to fit the sandwich, though some like the \tin{aesthetic} of the hanging
vegetable. butter the bread with rosemary hummus (see below) and place
the entire baked flower on your lover\index{love!lover} or customer's plate. it's good
for them.

\subsubsection{squash sandwich with carrot mayonnaise and a tray of roasted vegetables}

take onions and \gls{tranches} of butternut
squash and bake them in the oven with olive oil and black pepper and
rosemary (if you brought it from argentina). when huenu and muticia
handed you that bundle like a sword a season and climate ago, they
knew, yes they must have known, but how could they have known where it
would go and who it would be --- that their commonplace mother of
herbs would be \textit{the} exotic\index{exotic} touch in a land whose very name
instills throes of exotic\index{exotic} longing way down there in patagonia?

when they (the steaks, not your wistful longing --- there is a word
for that in portuguese --- for lovers\index{love!lover} or lovers-who-should-have-been. at the time it may have made some difference but now the looks
and the longing have settled together in a sweet smiling sorrow that
helps with the illusion\index{illusion} that yes you have lived) start to caramelize
you can take them out. i like to leave the peel of the squash on but
many humans are not used to such behavior. any other vegetables you
throw in with the broilers can be put on the side. this is the steak
of your sandwich

to doll up the orange\index{color!orange} meaty sweetness of the squash (or pumpkin or
whatever is in the family\index{family} at your local franchise organic\index{organic} farm store),
i like to throw in a few cloves and ginger and sunflower oil instead
of the above. the key magic ingredient to this --- which being from a
decidedly temperate fruit has neither right nor basis in an ostensibly
brasilian\index{brasil} cookbook -- is quince. the perfume of the baking quince does
a damn fine job of impregnating the squash. and you can use the baked
fruit afterwards for any number of sweet or savory treats.

for condiments i would butter the bread with a version of carrot
mayonnaise (see chapter five), and use thin slices of tomato and
cucumber on top. alternatively, you could place the steak on a layer
of torn spinach leaves (or other non-tasteless green, like mizuna,
endive, or arugula)

and of course in traditional brasilian\index{brasil} style, don't be afraid to toss
an oily fried egg on top before closing that sandwich.

\section{coconut sandwich loaves}

\begin{ingredients}
  \item water
  \item flour: half whole wheat and half white\index{color!white}
  \item coconut milk
  \item oil
  \item sugar
  \item salt
  \item yeast
\end{ingredients}

if you know how to make bread, just make bread with coconut milk
instead of butter or oil or milk or whatever you add to The Sponge to
soften the dough. if you don't, know that:

\begin{enumerate}

  \item bread is easy. bread is easy. bread is easy.

  \item it was probably a product of laziness and sloppiness by some
  ancient man who couldn't follow his woman's instructions,
  which is to say, it's available to all of us and should be an
  integral part of our lives. if you eat gluten and all of that.

  \item to make bread you need flour and water. that's all. to make
  tasty bread you also need salt. that's all. to make tasty bread
  and not have a live ETERNAL\index{eternal} BREAD CREATURE living in your house day
  and night, you'll need to buy some domestic yeast.

  \item there are exactly as many ways to make bread as there are
  names for the unnamable. don't let anyone tell you otherwise.

\end{enumerate}

on the eve of each day we opened the restaurant, i would mix my
yeasties in warm water with a few cups of flour. it formed a stringy
paste somewhere between what you would use to flier a city and what
your girlfriend's father might cook you for an awkward
breakfast. then i would go to sleep or walk on the beach or listen to
dj panozzo or have a few brasilian\index{brasil} beers or whatever it was and in the
morning the paste would have evolved into the beginnings of
bread. this first, overnight, rise was considered so important to
french bakers (back in the day, when people cared about themselves and
the world that touched them) that selling bread that had NOT risen at
least eight hours was a crime.

you will see bubbles of carbon dioxide throughout, feel powerful\index{powerful}
strands of gluten resisting your stirs, and smell the characteristic
\gls{cheiro} of yeasty living. this
paste is called The Sponge. add half a can of coconut milk to The
Sponge. at this point you may wonder, how much bread am i making
anyways? well, you can always control that by the amount of liquid you
initially add: \tin{XXXX} cups makes \tin{XXXX} loaves of bread.

after beating the \gls{leite de coco} well into The Sponge, begin to add your flours. mix in each cup
thoroughly and smoothly until you can no longer use a
utensil. continue to add flour, cup by cup, until you have a smooth
and slightly sticky ball, by Kneading together the dough with the raw
animal power of your hands.

the goal of the Kneading is to lengthen and to harmonize the stringy
tendons of the dough (called gluten) until you have a smooth rubbery
live ball. a popular technique is to push into the dough (and away
from your body) with the palm of your hand, stretching out the mass
only to pick up the end and fold it back to where you started. as you
begin to push again, turn the dough (or the entire bowl) ninety degree
so your next push is perpendicular. if you continue in this fashion
for 10 or 15 minutes your dough should be smooth, uniform, and
symmetrically developed.
	
it is extremely important, across cultures\index{culture} and epochs, only to mix the
dough with one hand. the ancients too understood that one must always
keep a hand clean and dry to turn the bowl, answer the phone, etc. so
much so that in argentina the expression for ``getting caught
red-handed\index{color!red}'' is ``getting caught with your hands in the dough''.

at this point the yeast need to frolic, yet again. protect the yeast
by oiling the bowl (to prevent sticking) and covering with a damp
cloth (to prevent drying). let them play until the bread has doubled
in volume; the carbon-dioxide has filled the cubby-holes of the gluten
lattice we constructed kneadingly. when the bread is large and
incharge, uncover and take a deep whiff. making bread is a meditation
and eventually we on this planet\index{planet}, together, will come to understand
that the making is, itself, the true reward. press down upon the bread
using your fist with a pressure correspondent to your mood. the spell
is broken, the carbon-dioxide escapes, and you are left with the
deflated doughy mess.

Knead again for a few minutes, oil gently and, cut the dough swiftly
and surely into as many loaves as you want (remember, they will grow)
and place in loaf pans to make sandwich bread. preheat the oven as
your bread rises one final time. know that you are preparing for the
wholesale slaughter of these creatures, for whom you have cared so
deeply and so handily. and yet, somehow, this is not only okay, but as
it should be.

when the oven is hot (say 350 F) enter the bread and do something
else. be nearby and check them when you smell them, or be somewhere
else and come back in 30-40 minutes. a few minutes after they look
done, verify. remove from the pans and tap the bottom --- a hollow
sound indicates the flour has absorbed the excess water and the
cooking is complete. turn off the oven and let the bread proudly rest
on top of your stove. do not attempt to cut and eat them immediately
as they will fall apart and burn your mouth.

\subsection{a gentle pattern to breadmaking}
	
there exist as many variations on bread as attempts to pronounce the
unspeakable.

\subsubsection{accepted ingredients}

\begin{ingredients}\item flour\end{ingredients}

i would prefer to use finely ground unbleached organic\index{organic} whole wheat
flour for one hundred percent of my baking needs. unfortunately, this
bread often doesn't rise (most of the whole wheat flour you get
isn't finely ground enough) to most people's satisfaction. and
we've grown up in a white\index{color!white} society in oh-so-many ways and maybe
it's not even desirable to totally turn our back on the wonderbread
fantasia of our collective past. so generally at O \tin{Bigode} we used
half/half whole wheat and white\index{color!white} flour, trucking the whole wheat flour
back from the store in Salvador and getting the
ultra-refined/bleached/processed white\index{color!white} flour from the 
\gls{mercadinho} in
Aratuba. in anglo-amerika, most co-ops have unrefined, unbleached,
white\index{color!white} flour which, while lacking the bran and germ of the wheat,
doesn't include the chlorine and whatever other spooky chemicals\index{chemical}
people get to taste in their wedding cakes.

\begin{ingredients}\item water\end{ingredients}

any water that's healthy to drink is healthy for bread, and i would
even use water that's not. you can boil water to stab out the
impurities and use it for The Sponge as it cools down to warm. a
popular measure for how hot is just right is the temperature your
wrist can handle comfortably, without focusing your attention on
trying to bend the Matrix.
	
remember that the amount of fluid in the bread (water + any fats or
specialties you use) essentially controls the volume of bread
you're making. the texture of the dough should be at least slightly
sticky, and perhaps even wet. wetter doughs are difficult to work with
and provide a moister final loaf.

\begin{ingredients}\item salt\end{ingredients}

salt, endorsed by french medieval bakers and neighborhood fatties
alike, brings out the flavor of the bread and retards the rising
time. it will kill the yeast if allowed to contact them directly (in
the case of dry, packaged yeast), so make sure to mix them in
solution. coarser salts tend to have more flavor and mineral nutrition
--- sea salt is nice, mountain salt is nicer.

\begin{ingredients}\item yeast\end{ingredients}

yeast are small fungi that are everywhere and alive. think of them as
microscopic everpresent friends from the mushroom kingdom, sent to
earth's airs to help humanity in nutrition and inebriation. we
evolved together. the same yeast that works with us to make bread
works with us to make \tin{alcohol}. historically, humans picked up on the
\tin{alcohol} first, began cultivating grain (millet) to get more of it, and
ended up with extra grain for less exciting culinary pursuits.

in the modern world it might help to think of yeast as alchemical\index{alchemical}
machines transmuting simple sugars to carbon dioxide and \tin{alcohol}. in
baking bread we're mainly concerned with the rising power of the
carbon dioxide; in making wine we're mainly concerned with the
liberating\index{liberation} power of the \tin{alcohol}.

yeast are like volcanoes in many ways. as far as i have noticed, they
exist among three main states of awareness --- death, dormancy, and
reproduction. they do not survive well in hot temperatures (hence
canning and pasteurization) and are made sleepy by cold temperatures
(so you can store them in the fridge). at room temperature and
slightly above they are happy and active, reproducing early and often,
giving off carbon dioxide and \tin{alcohol} to their fungal hearts'
content.

as with most things that are or were alive (like celery or chicken)
one can choose to buy them (usually dead) or to care for them
constantly and kill them ourselves. yeast are easier and cheaper to
care for than chickens and if you're going to make bread on a
regular basis it might behoove you to do so. all a yeast colony needs
to live is sugar. this can be processed sugar, raw sugar, fruit sugar,
or the sugar in flour.

\subsubsection{how do i create and care for a yeast colony (already in progress)?}

generally, selfishly, called a starter. that which starts your bread
also, somehow, ends their \tin{blockparty}. so be sure to give thanks.

mix a little yeast with warm water and flour and let it sit outside
(temperate or tropical zones, dry skies) for a few hours. it should
bubble and thicken. it is alive. store in your refrigerator to control
population growth and take out a couple spoons each time you make
bread. remember the words of meister eckhart ``that which we take in
through contemplation we must give out through love'' --- so be
sure to

\begin{enumerate}
  \item[a)] replace the colony you use with more yeast-food (some sugar,
  some flour)

  \item[b)] share your holy bread with lovers\index{love!lover} and strangers alike. (some
  mornings this will be easy)
\end{enumerate}

\subsubsection{what if i don't have any yeast to start with?}

you don't need to have yeast. the yeast have you. the sour smell of
your garbage. the ``rotting'' pear at the bottom of your fruit
bowl. the sediment in your microbrew. there are many more yeast than
there are humans and they live in most places we do. to entice/trap
some from the air simply leave some sugar/honey with water on the
table for a day or two. cover with a cheesecloth or clean t-shirt to
screen out flies and small children --- the pores will be enough for
invisible flying mushrooms to enter and to feast. when it starts to
bubble, add some more sugar and flour to the mixture and, voila, you
have your very own conspiratorial fungal colony.

\subsubsection{what about other ingredients?}

bread and beer purity laws, famous in france and germany, are the coy
straitjackets of poetic form. like haikus and sonnets, they impose
artificial limits in whose obedience we express ourselves and through
whose transgression we taste novelty and freedom\index{freedom}.

fats make your bread softer and spongier and should be used if
you're trying to appeal to non-french or non-ascetics. traditional
amerikan breads use milk, butter, or oil to soften their bread so it
lasts longer before hardening and is easier to chew. because i was in
brasil\index{brasil} and i'm not particularly fond of milk anyhow i used mainly
coconut milk for the fatty dimension. coconut palms are to brasil\index{brasil} what
cows are to india. and though the bread never tasted strongly of
coconut, i felt good knowing it was there.

sugars quicken the rising time as they provide more food and easier
access to the yeast. some of the sugar (from the flour and the
sweetener) will not be metabolized and this residue serves to sweeten
the bread. if you want your bread to taste store-bought, be sure to
add lots of fat, salt, and sugar, as those are the stalwart roots of
contemporary gastronomy\index{gastronomy}.

eggs are amazing and there's whole books about them so all i can
say is that if you're not opposed to it and you're using wheat
flour, the egg will really help in the rising action of the
bread. really, really help. beat it in to The Sponge with whatever
else you're adding.

goodies and knickknacks. you can add \tin{anything} you damn well please to
bread during the final rise. sesame seeds, nuts, fried onions, bacon
bits, cranberries, cinnamon, garlic, whatever. after you've punched
down the bread and before you separate it into loaves (or after if you
want some variety), take a handful of whatever you want to mix in and
lay on your floured surface. knead the bread into and around your pile
and eventually you will see little specks of pumpkin seeds or nutella
or nori showing up all over your bread.

\subsubsection{and if i live in a world with time constraints?}

there are those of us yet uncomfortable with manipulating space-time\index{space-time},
and it often happens that we decide we want bread in a couple of
hours, not across the long night of our future. hell, there's no
guarantee that either you or the sun will be here eight minutes from
now, much less tomorrow.

so yes, you can make bread without letting it rise overnight. you can
mix yeast and water in one bowl and flour and salt in another
one. when both are uniform add the flour to the water and knead
together until smooth. oil it well and let it rise in a warm place
(maybe a preheated oven to very very low). let it double in size,
punch it down, preheat the oven and let it rise again. when the oven
is hot you can bake it.

or if you're really in a hurry, forget about the second rise,
don't punch it down, and throw it (carefully) directly into the
oven. you will notice the difference in the texture of the bread, but
if you're hungry or craving in a profound way, you probably
won't care.

\subsection{variations on sandwich bread}

\subsubsection{cheddar / black pepper bread}

\begin{algorithm}
  \item make The Sponge
  \item add butter and a cup of milk or yogurt, mix well
  \item add a few tablespoons of freshly grated black pepper.
  \item make The Knead
  \item let rest, rise, and deflate
  \item Knead in a cup of grated cheddar cheese and some whole peppercorns
  \item form and bake
\end{algorithm}

\subsubsection{cranberry / walnut bread}

\begin{algorithm}
  \item make The Sponge
  \item add 1 cup of orange\index{color!orange} juice, a few tablespoons of maple syrup, mix well
  \item make The Knead, keeping it on the moister side
  \item let rest, rise, and deflate
  \item Knead in a cup of chopped walnuts and craisins
  \item form and bake
\end{algorithm}

\subsubsection{savory breakfast bread}

for eating with eggs, fried eggs.

\begin{algorithm}
  \item make The Sponge
  \item add olive oil, chopped saut\'{e}ed garlic and/or scallions, raw minced chives, freshly ground black pepper 
  \item make The Knead
  \item let rest, rise, and deflate
  \item Knead with a little more olive oil
  \item form and bake
\end{algorithm}

\section{brasilian\index{brasil} \gls{hummus}}

\begin{ingredients}
  \item a lot of cooked chickpeas
  \item a few healthy dollops of \gls{tahini}
  \item juice of a couple of passionfruits
  \item lime
  \item olive oil
  \item garlic
  \item roasted cumin seeds
  \item sal
\end{ingredients}

our brasilian\index{brasil} hummus was very similar to the traditional middle
eastern variety. the only ingredients in the lebanese hummus we used
to get were (in order of appearance)

\begin{ingredients}
  \item chick peas
  \item tahini
  \item lemon juice
  \item olive oil
  \item salt
\end{ingredients}

to this i like to add freshly ground roasted cumin seeds and roasted
garlic. since both are used extensively in bahian cuisine, they fit
with the motif and appellation. the real new world twist however comes
in the acid addition --- limes instead of lemons (so much easier to
find) and the alienheavenly bite of \useGlosentry{maracuja}{maracuj\'{a}}.

you'll probably never come across it unless you're buying food
to make this recipe (why?) or find yourself between henry's
tropics. but if you do, know that a good way to get the juice is to
blend the pulp, seeds and all, with just enough (clean) water, and
then strain through a tea or (more commonly available) juice
strainer. the black bits will stay behind and you can use the
(slightly) watered down suco de maracuj\'{a} as your lemon juice.

the key takeaway of this particular slide is not ``my food is so
much cooler because it's tropical'' or even that the taste walks
you over to the sublime (already in progress) but that anywhere you go
(more accurately, anywhere i've been) you can find the means to
produce the flavor sense combinations that simple good food
requires. hummus is about a creamy beany experience punctuated with
yelps of citric acid. you can get that acidity from vinegar (ouch),
raw mangos, sour oranges\index{color!orange}, gooseberries, grapefruit or even some types
of jawbreakers.

\subsection{varying the hummus \textit{pattern}}

as many intrepid cooks over the years have noticed, there are many
types of beans and all of them seem to fit pretty well in the \tin{mortar},
blender, \tin{robot}, etc.

the basic technique i've evolved into using follows:

\begin{enumerate}
  \item have your cooked beans ready to go. hopefully you pressure
  cooked\index{pcook} and didn't hurt anybody (see chapter zero).

  \item if you're using a blender or \tin{robot}, blend the beans with a
  little bit of the cooking liquid in the machine. they should blend
  smoothly but not too watery.

  \item when they've blended to just the right consistency,
  transfer to the mixing/serving dish and casually rinse the
  blender. since you didn't use any fiery spices or oils, it should
  easily rinse clear. this is important because blenders and robots\index{robot}
  can be a pain to clean if oily or left to crustify. also, with this
  technique you can quickly make three or four different
  ``hummus'' or ``bean p\^{a}t\'{e}s'' in parallel.

  \item add whatever spices, oils, vinegars, herbs, or vegetables you
  want and mix well.
\end{enumerate}

some thoughts regarding flavor combination and proportion:

\begin{enumerate}
  \item most hummus-type creations follow the model of bean / fat /
  acid / spices. beans are starchy and basic and will dry out (as well
  as ferment quicker) without the oily and acidic elements.

  \item you often don't need very much oil, there's a wide
  variety of happy middle ground, and putting too much will make it
  taste disgusting. recipes will always call for olive oil and you
  don't have to use it. mostly people who cook real, good, food in
  the world can't afford olive and they do without. it's fine.

  \item beans are large. don't be afraid to use strong spices in
  healthy amounts. garlic goes well with most beans. it's nice to
  use a fresh herb and dry spice in combination.
\end{enumerate}

all of these spreads are good with any kind of bread product, though
each has its traditional ally --- hummus with pita, black beans with
tortillas, canellini with toasted baguette slices.

\subsection{variations on brasilian\index{brasil} hummus}

\subsubsection{rosemary hummus}

\begin{ingredients}
  \item chick peas (see chapter zero)
  \item tahini
  \item oil olive
  \item lemon juice
  \item garlic
  \item rosemary
\end{ingredients}

this hummus variant uses an infused oil to incorporate the sweet
strong flavor of rosemary into the dish. while the chickpeas are still
cooking, place a couple cups of olive oil (or whatever oil you plan to
use) on low heat at the back of the stove and drop in a few branches
of rosemary. the oil will slowly heat and absorb the essence of
rosemary. let them cook together for about an hour, remove the
rosemary, and make the hummus as you normally would.

to give the garlic a subtler flavor add it to the infusing rosemary
for the last 10-15 minutes of cooking, then remove and chop or blend
as you would normally.

the herbal infusion can also be done without heating --- simply cover
a few branches of rosemary with olive oil in a bottle or jar and leave
in the sun for a week or two. you can do this with any and every herb
you find and build up an incredible collectors edition set of
beautiful thriftstore bottles each with a colored\index{color} infused oil or
vinegar. you'll be the talk of the town.

before serving, garnish with a mixture of finely diced onion, tomato,
and cooked rosemary leaves.

\subsubsection{split pea p\^{a}t\'{e}}

\begin{ingredients}
  \item one or two finely chopped onions
  \item caraway
  \item cooked split peas (see chapter one)
  \item some vegetable oil
  \item fresh parsley
  \item lemon juice
  \item sal
\end{ingredients}

this follows the typical model except the onions it incorporates are
fried separately to give a deep taste and crispy texture. lay a pan on
high heat, add enough oil to thinly cover its bottom, and add the
onions when hot. you'll have to be attentive and stir constantly to
fry well without burning. two onions will likely give you much more
fried onion than you need but everyone loves fried onions so don't
sweat it. when the onions are fully brown, turn off the heat, give a
final couple of stirs, and incorporate into the p\^{a}t\'{e}. if you
prefer some spice, add cayenne pepper to the frying onions: late in
the game but a few minutes before the end so their essence can mingle
and develop.

the caraway seeds, toasted and ground, provide an earthy flavor that
the beans desperately lack. if it ends up tasting like nothing,
it's because there are too many peas for the onions and caraway. if
you don't have caraway, use cumin, fenugreek, or even dill.

the parsley is finely chopped and gets folded in at the end with the
salt and lemon juice.

\subsubsection{black bean dip}

black beans are generally acknowledged to be ``the'' favorite
bean, so of course there are many ways to enjoy them. if you have
leftover black bean soup or refried black beans, pounding or blending
them into a dip is the natural next step in the evolution\index{evolution} of your (you
and the beans') relationship. unflavored black beans work as well
but would benefit from the treatment given to the canellini beans,
below.

\begin{ingredients}
  \item black beans (see chapter zero)
  \item some cooking oil
  \item garlic
  \item hot chiles
  \item ground roasted cumin
  \item raw chopped onion
  \item salt
\end{ingredients}

it's generally more appealing with some whole beans intact, so i
generally mash the p\^{a}t\'{e} by hand instead of using the \tin{robot}. the
garlic and chile should be diced (or ground) together and mixed in
from the beginning, so they have the most time to disperse. i add the
onion last so it retains both shape and pungency. it's nice to
serve a little chopped tomato and fragrant ground roasted cumin on
top.

\subsubsection{white\index{color!white} canellini bean spread \useGlosentry{aux herbes fraiches}{aux herbes fra\^{i}ches}}

this spread differs from most others in that it requires a minor
``refrying'' of the beans to get the right consistency and flavor.

\begin{ingredients}
  \item garlic
  \item rosemary, lavender, oregano
  \item canellini beans (see chapter zero)
  \item salt
  \item black pepper
  \item olive oil
\end{ingredients}

heat a couple of tablespoons of olive oil in a pan and saut\'{e} the
garlic on medium heat. chop the herbs --- use only what you have,
every incarnation will be different and somehow equally part of the
majesty of The Creation. when the garlic begins to brown and strongly
flirt with fragrance\index{fragrance}, add the beans and herbs. stir and mash in a
sweeping rounded motion until you have a white\index{color!white} spread punctuated with
bits of green and brown. some of the beans should be intact and
provide a textural experience.

salt and pepper to taste; the final product should be creamy with a
\tin{balance} of olive oil, garlic, and rosemary.

\section{a mediterranean salad, \useGlosentry{a la tropicale}{\`{a} la tropicale}}

the standard amerikan idea of ``salad'' involves lettuce, whereas
i've found the typical other-world idea of ``salad'' involves
any kind of random vegetable --- often even cooked. culture\index{culture} from
bulgaria to lebanon knows a salad heavily based on tomatoes,
cucumbers, and some sort of feta cheese. feta cheese and good olives
are available to the 1\% of the brasilian\index{brasil} population who know about
their existence, dare to enter the upper-class supermarkets, and have
the financial huskies to pull that sleigh. most of the time that
wasn't our scene but in the final analysis, good olives are good
olives.

\begin{ingredients}
  \item a couple of ripe red\index{color!red} tomatoes, sliced radially into wedges and
  then again in half

  \item one cucumber, peeled, seeded, and roughly diced

  \item olives and their juice

  \item crumbled feta cheese

  \item \onehalf a white\index{color!white} onion, roughly but thinly sliced

  \item olive oil

  \item lemon juice

  \item salt

  \item freshly ground black pepper
\end{ingredients}

i would cut the onion first and let it sit the lemon juice (we generally used 
lime) for as long as possible, to en-suave its sharp edges. failing lemon 
juice, wine vinegar works well. then i'd chop everything\index{everything} else and mix together. 
the olive oil is generally a strong presence, though not sickly dominant. as 
with most salads without crisp greens, this benefits from a few hours of 
togetherly marination.

\subsection{Variations on longing for the mediterranean pattern}

each micro-clime will have its own variations of course, but most seem to be 
based in tomato, cucumber, or both. the presence or absence of the local 
feta-type cheese is an important distinction, as well as the types of olives. 
fresh herbs, such as sage or thyme, add a nice dimension, as does minced garlic 
(either alone or with some olives, creating a partial \gls{tapenade} vibe).

some cultures\index{culture} prefer to base the salad on chopped fragrant herbs such as 
parsley, cilantro, or mint. you can add equal parts of all three, finely or 
roughly chopped, to the standard version for a powerful\index{powerful} and stimulating dish.

this whole school of salad can also be made with a cooled cooked grain as a 
primary ingredient. bulgur and couscous are used often in the middle east to 
provide starch, and texture. simply cook the grain as directed (see chapter 
six), cool, and mix in to the salad.

if you have easy access to it (like a tree in your back yard), avocado and 
coconut make great tropical additions to this salad family\index{family}. i would toss 
avocado with a little lemon juice and add it at the end to avoid crumbling. 
fresh or dried coconut should be grated and toasted (very very carefully) and 
sprinkled on top.

\subsection{variations on a mediterranean salad}

\subsubsection{lebanese fresh \gls{zaatar}}

\begin{ingredients}
  \item tomatoes
  \item cucumber
  \item olives
  \item onion
  \item lots of lemon juice
  \item lots of olive oil
\end{ingredients}

and for the spices, equal quantities of:

\begin{ingredients}
  \item fresh thyme leaves
  \item sesame seeds
  \item ground sumac
\end{ingredients}

\subsubsection{couscous salad}

\begin{ingredients}
  \item a couple cups of loose couscous, cooked in some coconut milk (see chapter six)
  \item diced cucumber or slightly underripe mango
  \item minced onion and garlic
  \item lime juice
  \item a pinch of cinnamon
  \item a handful of raisins
  \item carefully toasted pinenuts
\end{ingredients}

\subsubsection{tabbouleh}

\begin{ingredients}
  \item a couple cups of cooked bulgur (see chapter six)
  \item finely chopped parsley and cilantro
  \item diced tomato and cucumber
  \item minced onion and garlic
  \item lemon juice
  \item olive oil
  \item salt
  \item freshly ground black pepper
\end{ingredients}

\subsubsection{serve yourself}

in brasil\index{brasil} and argentina, salad often meant a few distinct piles of vegetables 
on different parts of a large plate, include boiled potato and cauliflower, 
huge slices of underripe tomatoes, a small dish of olives, and some tasteless 
lettuce. let that be your inspiration.

the plus side of the \`{a} la carte salad? it's a culinary structure designed 
to empower\index{empower} each diner to take control of her own gastronomic\index{gastronomy} experience --- we 
can all make our own mixture from the large-cut vegetables on a centralized 
plate. white\index{color!white} wine vinegar, olive oil, and salt and in ready reach for each 
person to dress their own

refrain from mixing together:

\begin{ingredients}
  \item large tomato slices
  \item cucumber rounds
  \item diced sweet peppers
  \item onions in half moons
  \item a pile of crumbled feta
\end{ingredients}

and a few doves, parsley- and olive-branches to decorate.
