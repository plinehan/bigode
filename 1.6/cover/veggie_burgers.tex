\mychapter
{chapter_corn.jpg}
{veggie burgers . buns . ratatouille . carrot mayonnaise . condiments}
{black bean veggie burgers between half wheat burger buns with
ratatouille, carrot mayonnaise, and a host of the standard amerikan
condiments}

\section{black bean veggie burgers}

so the burger has been long since liberated\index{liberation} from pigs or any of that hambu(r)g 
(ding!) and once we see things As They Really Are, of course you could mash up 
\tin{anything} under god's green sun, find a binder, and grill it. since i'm still 
abiding the illusion\index{illusion} that protein Is and Is Needed by the body, O \tin{Bigode} 
veggie burgers were always bean-based.

a veggie-burger being little more than ground up beans and spices, the key 
technical difficulty comes in forming the patties. if you and i are indeed the 
same person\index{same person} and you're hesitant to use eggs (out of \tin{ahimsa} perhaps) or flour 
(because it sticks to your belly perhaps) to bind your burgers, you need to 
have the moisture content exactly right. too much or too little water and they 
won't hold together well or fall apart during cooking. again, anybody who 
gives you a recipe and tells you it's the Tao is leading you astray --- there 
is only one unique moment for where you live and the humidity in the air and 
the temperament of your oven and how much water you really did cook out of 
those carrots.

\begin{ingredients}
  \item 1 lb dry black beans (should make a dozen burgers)
  \item one large onion, diced
  \item one large carrot, grated
  \item a few cloves of garlic
  \item a knob of ginger
  \item cumin
  \item oregano
  \item cayenne pepper
  \item salt
\end{ingredients}

\tin{pcook} the beans as directed (see chapter zero) and drain. if the beans are 
cooked too much and feel soggy it will be harder to form your patties.

saut\'{e} onions in a little oil with the spices above. fry them past 
translucent, releasing all their water and developing together with the cumin 
and oregano. you can either add the garlic and ginger now (to cook), or mix 
them in raw in the next step, depending on how strong you want their flavor.

when the onions are browned and fragrant, add the carrot and cook for a few 
minutes, expelling the carrot's pent up liquid and frustration. when your 
mixture is sufficiently dry, add it to the beans and mix well with your hand 
or a couple of forks. the starch from the beans should coat everything\index{everything} 
together and allow you to make individual patties. before starting, taste the 
mixture for salt and \tin{balance} and make any final adjustments.

if you can't make patties which hold together well without sticking all over 
you and everything\index{everything} else, you may want to consider a binder. effective 
ingredients include:

\begin{ingredients}
  \item an egg, lightly beaten
  \item wheat flour, start with half a cup
  \item chickpea flour, start with half a cup
\end{ingredients}

i usually used wheat flour, but not too much of it, if necessary. after making 
the patties and placing them on an oiled dish, i would chill the burgers for a 
few minutes in the fridge to help them hold together. chilling too much causes 
dryness and cracking, so, as in all things, strive for \tin{balance}.

bake the burgers on a medium oven heat (say 350 F or 180 C) for fifteen 
minutes on one side, then flip and finish for ten on the other side. check 
frequently to make sure they aren't drying out too much --- the surface should 
be firm and crispy but not cracking. since everything\index{everything} is already cooked, 
you're primarily concerned about heat, texture, and togetherness rather than 
whether or not it's raw in the middle.

black bean veggie burgers are best with fresh buns, fresh mayonnaise, crisp 
lettuce or spinach, and a juicy hunk of tomato.

\subsection{the \textit{pattern} to veggie burgers}

so veggie burgers can be made out of any bean you have available, following 
the general process below

\begin{algorithm}
  \item \tin{pcook} and drain the beans

  \item choose whatever vegetables you want to accompany them (onions, 
  carrots, celery, leaks, squash) and saut\'{e} the ambient liquid away

  \item elect any spices you might want (garlic, chiles, toasted cumin, herbs)

  \item mix the beans, vegetables, and spices together at room temperature

  \item form patties, place on a pan, and bake
\end{algorithm}

the smaller the patties are in diameter, the easier formation will be. the 
thicker they are the better they will hold together and the harder they will 
be to eat. getting a wide thin patty (to resemble the frozen ones you've seen 
all your life) is the hardest because the beans and vegetables have no glue as 
strong as the meat product floating out there in the world. flour and/or egg 
make the binding a lot easier, and adding a little oil to the mixture helps as 
well.

\subsection{variations on black bean veggie burgers}

\subsubsection{black-eyed fenugreek burgers}

somewhere along the long and winding road i became obsessed with the 
combination of feij\~{a}o fradinho and ground roasted fenugreek i had brought 
from india. for the first few months i couldn't figure out how to use the 
impossibly hard seeds and the only recipe i had for them involved sprouting. 
of course at some point she who discovered fire passed the secret on to me and 
i found that ground, roasted fenugreek had a rich, red, earthy, bitter taste 
that was perfect for one of bahia's favorite feij\~{a}o.

\begin{ingredients}
  \item 1 lb dry black-eyed peas; soaked, pcooked\index{pcook} and drained
  \item 1 tablespoon fenugreek seeds; roasted and ground
  \item 2 onions, diced and fried
  \item 1 bunch of green onions, diced
  \item a few cloves of garlic, minced into a paste
  \item salt
\end{ingredients}

mix together, form patties, and place in dish coated with azeite-de-dend\^{e}. 
the dend\^{e}-fradinho combination is celebrated by baianos, and i trust them.

\subsubsection{falafel}

falafel is the middle-eastern veggie burger of choice. i've had it made from 
chickpeas, fava beans, and lentils, sometimes with bulgur and sometimes 
without. there are two ways to make it:

\begin{algorithm}
  \item as above
  \begin{itemize}
    \item use 3:1 ratio of chickpeas to bulgur (see chapter six)
    \item mix in garlic, lemon peel, cumin, finely chopped onion
    \item chop together a bunch each of cilantro and parsley
    \item squeeze out the water (very important) with a towel and only then add to your sacred mixture.
  \end{itemize}
  	
  \item as below
  \begin{itemize}
    \item soak the chick peas overnight, rinse, drain, and blend thoroughly, adding water as needed for lubrication
    \item mix in spices and cooked bulgur as above
  \end{itemize}

\end{algorithm}

after blending the chickpeas may need to strain out some more water, (use a 
t-shirt, cheesecloth, or pillowcase) if your \tin{robot} isn't very professional. 
either way, falafel is generally made into smaller balls (closer to flattened 
ping-pong balls than burgers) and deep fried. eat it with hummus and tabbouleh 
and \gls{labneh} in a pita. 
\gls{humdulillah}\index{humdulillah}.

\section{half wheat burger buns}

making buns is essentially the same as making bread. i add more fat (leite de 
coco) for more sponginess, and make some obvious changes in the shaping stage. 
the buns will be smaller, rise more, and cook more quickly than your typical 
loaves.

use (unbleached) white\index{color!white} flour for The Sponge and mix in

\begin{ingredients}
  \item vegetable oil
  \item a little honey
  \item sesame seeds
  \item sunflower seeds
  \item black pepper
  \item a little salt
\end{ingredients}

the following morning.

when you tighten the dough up for The Kneading, start with wheat flour and 
gradually switch to adding both as the proportions even out.
when you partition the dough for cooking, remember that each bun will lazily 
expand to fill any empty space in the pan, so pack them tightly.

\section{carrot mayo}

\begin{ingredients}
  \item just enough cooked carrots
  \item 3-4 inches ginger
  \item soy sauce
  \item rice wine vinegar
  \item brown sugar
  \item cilantro garnish
  \item toasted ground sesame seeds
\end{ingredients}

carrot dressing. carrot mayonnaise. carrot butter. carrot p\^{a}t\'{e}. it's 
easy, versatile, and yet shockingly ``original'' to most of your friends' 
parents. if you want the taste to back up the color's\index{color} promise, use carrots 
that DON'T taste like water.

the best cooking method is to saut\'{e} the carrots and ginger together in 
vegetable oil. the thinner and more evenly you slice them, the quicker they 
will cook. use a medium-high heat and stir frequently. as they get tender 
enough to mash or blend, throw in a tablespoon of brown sugar to provide a 
glaze. then blender them all save the two nicest-glazed slices, reserving for 
a garnish. 

add small amounts of your liquids to help the carrots pur\'{e}e. i generally 
use a small amount of rice wine vinegar, more soy sauce, more sesame oil, and 
more water. when they blend easily add the ground sesame seeds and taste for 
sweetness and salt.

% keep "pour" on the same page as the rest of the paragraph.
\begin{minipage}{\textwidth}
at this point ONLY YOU CAN DECIDE the consistency. for a mayonnaise-type 
sandwich spread, it's probably perfect as is. for use on a salad you'll want 
to dilute with salad oil and water to be able to\\
p\\
\makebox[1ex]{}o\\
\makebox[1.5ex]{}u\\
\makebox[1.8ex]{}r
\end{minipage}

\subsection{carrot mayonnaise \textit{pattern} speaks}

the elemental idea of this pattern of sauces is that a smooth blend of a tasty 
vegetable with light seasoning makes an excellent condiment, running the 
gauntlet from marinade to dressing to salsa to mayonnaise to dip. it's the 
same idea as a pesto except utilizing the world of barely cooked vegetables 
instead of raw greens. you can make such a p\^{a}t\'{e} from any vegetable you 
want. the only key is not to overcook the vegetable in question --- after a 
few tries the spices will suggest themselves.

\begin{algorithm}
  \item buy a vegetable you've never had before. peel it for good measure (how 
  are you supposed to know?), chop it into bite-sized pieces, and steam it for 
  five minutes. try a piece. if it tastes too hard, raw, or otherwise 
  poisonous, continue steaming for another five minutes and try again. when 
  you think the vegetable is ready for human consumption, drop a few specks of 
  salt on a piece and try it. do you like it? yes, yes you do. fresh 
  vegetables are good and salt is good and whenever the two meet, only 
  goodness is to be met upon.

  \item colocate the unknown steamed vegetable into the \tin{robot}. process it, 
  adding a little bit of oil if necessary for lubrication. if the vegetable 
  has a strong flavor, use a mild oil (canola, peanut); if it has a mild 
  flavor use olive  or a smidgen of sesame. now add a little salt and taste 
  it. what does it need? would it taste good with garlic? with lemon? with 
  something toasted? with tomatoes? with raw onions? with caramelized onions? 
  with orange\index{color!orange} juice? with apple juice?

  \item open your fridge and assess its contents. look back and forth 
  repeatedly between the mess in the coldbox and the paste on the counter. 
  with whom would what jive? this shaking of the head is the essence of 
  improvisation --- a dance between getting what you've always wanted and 
  getting rid of what you already have.

\end{algorithm}

\subsection{variations on carrot mayonnaise}

\subsubsection{beet mayo}

\begin{ingredients}
  \item 2 large beets
  \item olive or salad oil
  \item 1 tsp ground cloves
  \item 1 cm ginger
  \item cilantro
  \item 1 lime
\end{ingredients}

the technique here is basically identical to the carrot pur\'{e}e above. the 
spices are of course different, and i prefer the boiling method for the beets 
because of the nutritious tea / broth.

wash the beets well and peel if necessary (if they were venomously grown or 
have recalcitrant wa(n)ds of dirt). slice into chunky quarters and throw into 
a pot of boiling water. when they are tender enough to boil strain out of the 
pot and into the blender. SAVE THE WATER.

you can use the water for soup stock or to cook pasta, rice, or beans. you can 
drink it as a tea or chill it for an iced tea. it's very good and good for 
you. if you drink enough of it your pee will turn a reddish purple{color!purple} and really 
that's the best part.

blend the beets with enough of the reserved water to form a pur\'{e}e. remove 
from the blender (it rinses easily; water soluble!) and mix in some oil, the 
ground cloves, minced ginger, chopped cilantro, and lime juice.

you don't need a lot of beets to make a good amount of dressing. but as long 
as you're boiling beets, you might as well boil a few more to use chilled in 
salads later on. when the afternoon hits and you lose count of the beers, you 
can pull out a tray of cool vinegary beets and it will be goddamn\index{god!damn} perfect.

and a tip for boogie-impressive cooking: get beets of different colors\index{color} (some 
purple{color!purple}, some golden, some chiogga) to make multiple versions of this and place 
side by side. they don't even have to be spiced differently --- just put them 
on the same dish in yin-yang or checkerboard pattern and even the staid will 
cartwheel across your placemats.


\section{vegetable ratatouille}

we seldom had good eggplants at O \tin{Bigode} and when we did i always made 
ratatouille. it's a tomato-based stew flavored with mediterranean herbs, 
popular in the south of france and generally served over couscous. i also 
like it as an appetizer with pita bread or chips and cold on sandwiches the 
next day.

\begin{itemize}
  \item one large, bulbous, naked aubergine
  \item one long summer squash or courgette
  \item a few red\index{color!red} and green bell peppers
  \item some onions
  \item lots of garlic
  \item oldish tomatoes, some cans of stewed tomatoes, or tomato sauce.
  \item herbes de provence (lavender, rosemary, sage, oregano, etc...)
  \item cayenne pepper or paprika
\end{itemize}

dice the eggplant into largish cubes and mix in a bowl with a few teaspoons of 
salt. set aside and do not disturb --- the salt will wage war on the 
eggplant's hidden bitter juices, drawing them out of their caves to be rinsed 
away.

dice all your peppers and enough onions to have an equal quantity. cut the 
courgette into rounds or halfmoons. mince the garlic.

heat some olive oil in a pan and saut\'{e} as directed above (see chapter 
three). start with the onions and peppers, adding the garlic and red\index{color!red} spice 
when translucent. after you add the garlic, rinse the eggplant and wring any 
remaining water out of the cubes. the salt has drawn the bitter spirits to the 
surface (in solution) and it's up to our heroes to vanquish them.

you'll finish as the garlic browns and all timing is perfect for you to add 
the \gls{courgette} and \gls{aubergine}. continue to saut\'{e} until all 
the vegetables are just soft, and as their released water steams away, add 
whatever combination of tomato essence (chopped real tomatoes, canned 
tomatoes, tomato sauce, ketchup) you have on hand. dump in whichever herbes 
you have, turn the heat to low, and let their power open up your mind. focus 
your mind, dilate your pupils, and notice your saut\'{e} evolving into a stew 
--- the tomatoes' juices seep and comfort on every side, the \gls{bouquet des herbes} slowly diffuses into every liquid corner, and the Many meld into the One\index{One!and the Many}. 

you'll have to simmer for at least half an hour to thicken and cook out any 
acidic weirdness from the tomatoes and their cans. parsley and a considered 
teepee of finely grated hard cheese make a great garnish.

\subsection{the \textit{pattern} to vegetable ratatouille}

ratatouille is basically a stew, and a stew --- \textit{\gls{selon moi}} --- is basically 
saut\'{e}ed vegetables to whom are added liquid and more vegetables. 
ratatouille variations might hold the herbes de provence and tomato base 
constant while changing which vegetables comprise the bulk of the stew, which 
herbs are given more weight, and how spicy the end result is. besides tomato, 
coconut and peanut make good stewing mediums, as do starchy vegetables like 
potato or mandioca which naturally pur\'{e}e into the mix. 

\subsubsection{variations on herbes de provence}

herbes de provence is a fragrant herb combo sold in cute yellow\index{color!yellow} cloth bags in 
the south of france. it's a marked-up mixture of all the aromatic weeds native 
to provence, which might include thyme, marjoram, oregano, basil, summer 
savory, lavender, rosemary, sage, and occasionally fennel. if i don't have the 
name-brand variety i'll try to rob at least the sage, rosemary, and lavender 
from neighboring gardens, though the fennel adds a nice touch that can't 
easily be replicated (think pastis).


\subsection{variations on vegetable ratatouille}

\subsubsection{west-african style peanut stew}

saut\'{e} green peppers and onions first, like in ratatouille; add the 
following when translucent:

\begin{ingredients}
  \item a couple of small hot chiles
  \item a chunk of minced ginger
  \item SOME MYSTERY AFRICAN HERB
\end{ingredients}
	
brown for a few minutes and add some sliced carrots (or other rooty veg) to 
the saut\'{e}. cook for a couple of minutes before adding a mixture of peanut 
butter, water, and chopped tomatoes (combined like in chapter three) to turn 
your saut\'{e} into a stew. turn the heat down to medium-low, cover, and 
simmer until everything\index{everything} is cooked through. if the soup is too watery, take off 
the lid to boil off some excess liquid.

salt and pepper to taste.

\subsubsection{xu-xu moqueca}

instead of the full-blown version (see chapter nine), sometimes i would throw 
together a quick xu-xu version to give foreign visitors a taste of this 
strange vegetable. it might be called prickly pear in english (or maybe that's 
something else) --- a spiny light-green affair whose flesh inspires puffiness 
and a mild rash on those who dare to cut it. despite its ornery nature, i 
loved it steamed, saut\'{e}ed, battered, and stewed.

if you can find it, be sure to peel the xu-xu well and remove the inner seed 
area. peeling it under running water helps with the rash as well but i 
strongly recommend getting the full-blown rash a couple of times for the 
experience. the puffy disconnection your hands undergo is not entirely 
unpleasant, and gives an immediate material reminder that we live in a 
complicated and dangerous wor(l)d.

if you don't happen to see it lurking nearby, any vegetable on the harder side 
(like a carrot or a potato) would work well in this recipe.

\begin{ingredients}
  \item a few xu-xu
  \item a large onion
  \item a few cloves of garlic
  \item a teaspoon or two of cumin
  \item two chopped tomatoes
  \item one can of coconut milk
  \item chopped parsley
\end{ingredients}

real brasilian\index{brasil} moqueca uses a special palm oil called `dend\^{e}'. if you have 
it, use it and admire the deep red\index{color!red} color and powerful\index{powerful} fragrance\index{fragrance}. if you don't, 
any palm oil will have familial affinity, and if not, use peanut oil to get 
some sense of how fatty it should be.

saut\'{e} onion and garlic together on medium heat in your elected oil until 
translucent. add cumin and your thin spears of xu-xu (or whatever you chose 
for dinner) and stirfry for a few minutes. when the cumin's color\index{color} and scent 
fill the kitchen, add tomato and coconut milk. stir well and lower the heat to 
medium-low --- continue cooking until the vegetables are as you like them. 
salt and garnish liberally with chopped parsley. the color\index{color} should be a deep 
oily red-orange\index{color!orange} with the dend\^{e} oil, and a pinker shade (somewhere between 
tomato and coconut) without.

\section{standard amerikan condiments}

ketchup, mustard, and mayonnaise. if you're living in brasil\index{brasil} (and loving it) 
and yet need a nostalgic (the brasilians\index{brasil} have a word for that) burger 
experience, you must synthesize amerikana out of (perhaps healthier) basic 
reagents. luckily for the culinary tripper, all three are a) pretty easy to 
make yourself, b) much better when so prepared.

\subsubsection{ketchup}

ketchup is an industrially produced reduction of rotting tomatoes with enough 
salt and sugar to disguise any real vegetative taste or texture. as such, a 
blender is helpful. take all the old tomatoes you can find or buy a few cans 
of tomatoes or unadulterated tomato sauce (in brasil\index{brasil}, it's impossible to buy 
boxes/cans of tomatoes without sugar) and cook them over low heat for at least 
an hour on the stove. watch the volume reduce and thicken into a paste. when 
sufficiently thick add salt and sugar, stir for a few minutes, and transfer to 
the blender for final processing. start with a small amount of sugar, perhaps 
5\% by volume of the tomato sauce. continue to blend and taste until it begins 
to taste like ``real'' (fake) ketchup. stop when you reach your goal or are 
sickened by the quantity of sugar it involves.

if you want your ketchup to have some sort of flavor (maybe like barbecue 
sauce) you can roast the tomatoes beforehand and blend in roasted (or raw) 
garlic and onions. cumin too will make it taste more like adult food, but know 
that's against the basic premise.

\subsubsection{mayonnaise}

as far i can read on the label, typical mayonnaise has no relation to the 
traditional french process, which is fatty, delicious, and even slightly 
dangerous (oh my!) . french mayonnaise (called aioli when spiced up a little) 
is a carefully bound mixture of eggs, oil, and lemon juice. beat 1 egg for 
each cup of oil you're going to use and place in the jar of your blender. 
while running the machine on low speed, slowly drip in your cup of oil. if you 
drip it in slowly enough the oil will incorporate into the eggs in a beautiful 
demonstration of COSMIC\index{cosmic} UNITY\index{unity} and no separation will occur. add two 
tablespoons of lemon juice per oil-cup and continue with the oil, adding until 
your cup runneth empty or your mayonnaise has achieved the consistency you're 
looking for. salt. it'll be the best damn mayonnaise you've ever had and won't 
last long due to human and bacterial influences.

to make aioli, or its fancy (tarragon aioli, saffron aioli) counterparts, 
simply blend the eggs in the beginning with garlic and whatever herbs and 
spices you'd like to experience. use nicer oil. fresh herbs (chives, tarragon, 
parsley) work marvelously, especially the proven\c{c}al collective (lavender, 
rosemary, sage, oregano). know you are a samurai and that most restaurant 
``aiolis'' are made by blending food coloring\index{color} with huge vats of jiggling 
hydrogenated mayonnaise.

\subsubsection{mustard}

mustard again is a simple construction made from vinegar and the mustard seed. 
the homemade variety i like gives an ``old-world'' style mustard that is 
nowhere near as creamy nor fluorescent as your standard variety. simply blend 
equal quantities by volume of mustard seeds and vinegar in a \tin{robot} with salt, 
peppercorns, and whatever other spices come to mind. let it age a week or so 
to work over the bitterness and initial pungency into a suaver taste.

for 1 cup of mustard seeds and 1 cup of vinegar, you could also add:

\begin{ingredients}
  \item a whole head of garlic
  \item \threequarters cup of honey (honey-mustard, anyone?)
  \item \onequarter cup of horseradish
\end{ingredients}

remember that you can use vinegar already infused (with garlic, herbs, ginger, 
whatever) to get a jump on flavoring your mustard.
