\mychapter
{chapter_carrots.jpg}
{chana masala . chapattis . cilantro-peanut chutney . carrot salad . raitu}
{a typical chana masala with whole wheat chapattis,
ringed by cilantro-peanut chutney, gujarati carrot salad, and cucumber
raitu}

\section{\gls{chana masala}}

chana masala is a punjabi-style curry popular in restaurants all over north 
india and the west. it was the first dish my mother taught me to make and the 
most-requested item in my repertoire.

you will need

\begin{ingredients}
  \item cumin seeds and black mustard seeds
\end{ingredients}

a good note on improvisation:

one general rule i cook by is that if i can't find the ingredient listed or 
don't want to be pushed into using it, i'll try to substitute naturally with 
something similar. often, especially with rare, expensive, or boogie items, it 
helps to arbitrarily strike words off of the ingredient until you find 
something usable.

for example if you don't have ``black mustard seeds'' try for just ``mustard 
seeds''. if you don't have ``cumin seeds'' you can try using just ``cumin'' or 
maybe some other ``seed''. what's the worst thing that could happen? ground 
cumin toasts quicker than whole cumin seed so you have to be more attentive. 
using another edible spice seed would give a different flavor and you'd end up 
with a different dish if you were to use anis or caraway. you might even win a 
fusion prize or something.

\begin{ingredients}
  \item \onethird -\onehalf as many chopped onions as cooked chickpeas, by volume.
  \item 1-2 cloves of garlic for each medium-sized onion, and equal amount of ginger and hot chiles
  \item cooked chickpeas
  \item a couple of tomatoes
  \item some chopped cilantro
\end{ingredients}

and the following ratio of spices:

\begin{ingredients}
  \item 1	salt
  \item 1.5	coriander
  \item 1	cumin
  \item 1	garam masala
  \item \onequarter	turmeric
  \item \onehalf	cayenne
\end{ingredients}

to make the curry, high heat enough oil in a pan to cover the bottom when hot. 
if you like that delicious fatty taste use ghee instead of oil and use twice as 
much as you really need. there's some truth to the argument, ``hey, i'm 
vegetarian, i can eat as much fat as i goddamn\index{god!damn} please'' unless you're talking 
to a vegan.

when the oil is hot, add the seeds and have a lid handy to cover. they will 
begin to pop fragrantly all over your kitchen. when you hear the popping stop, 
immediately toss in the onions. they should be chopped to a carpal's length. 
failing to do this immediately will result in the post-popping burning of your 
mustard seeds.

the onions are now frying in the infused oil, and you will agitate them 
vigorously until they brown. past saut\'{e}, past translucent, past browning, 
to brown. the color\index{color} will actually change to brown. toss in the ``wet spices'': 
the ginger, garlic, and green chile you chopped together. fry together with the 
onions for a minute until they too have browned. the essence of the curry now 
awaits the dry spices, which provide both the musky depths of flavor and the 
signature color\index{color}. sprinkle the mixture of spices (above, your homemade curry 
powder) over the frying onions and stir to incorporate. if you don't stir well 
they will burn before they brown --- the key is to heat the spices enough to 
release their flavors into the oil and the air. as you continue to cook and 
stir the fragrance\index{fragrance} will continue to get stronger and the curry itself will 
start to stick.

persist.

only when it is sticking too much to stir add the chickpeas and lower the heat. 
if you do it too early the flavors will not have had time to develop and you 
will not be satisfied with the profits of your play. let the chickpeas visit 
and curry for a few minutes then add the chopped tomato.

my mother adds tomato paste and sugar as well, and her curry is certainly 
better than mine. we both serve it with cilantro.

\subsection{the \textit{pattern}: ``how to curry \tin{anything} you damn well please''}

chana masala is an example of a ``punjabi-style'' curry made using onions as 
the vehicle to carry the flavor of dry spices through the curry. i'll outline 
below the general Form of all such curries. know that there are a variety of 
other forms, and you will not be able to make all curries with this Form, 
though the number of curries you can make is impossibly large.

\begin{algorithm}
  \item heat your cooking oil (mustard traditionally, canola usually, peanut or 
  soy if it's all you have). in brasil\index{brasil} they sold non-gmo soybean oil at the 
  local market and we would usually use that.

  \item pop some spices in the oil to infuse it.
  
  black mustard seeds, cumin seeds, fenugreek seeds, and dried red\index{color!red} chiles are 
  all popular choices. mustard and red\index{color!red} chile often go together.

  taking the curry in a sweeter direction, you can use a combination of 
  cardamom pods, clove buds, cinnamon bark, and whole black peppercorns (this, 
  roasted and ground together, is most of the mysterious garam masala).

  \item after the seeds sizzle and pop, the flavored oil is ready to accept the 
  main taste vehicle: onions
  
  dice onions evenly but not obsessively small. i use about half an onion per 
  person i expect will be eating.

  stir the onions well at first, to coat thinly with oil and spices. fry past 
  the soft translucent stage until they brown.

  the onions will turn brown and let the oil they've absorbed back out into the 
  wild of the pan.

  \item add your wet spices 

  the wet spices, known alternately as ``fresh'' or ``green'' spices, are 
  generally garlic, ginger, and green (or any color\index{color}) hot chiles. depending on 
  what you're going to curry, you will add somewhere between all and none of 
  these. chop them well or mash them together in your \tin{mortar} and fry for a 
  couple of minutes before continuing.

  \item add your dry spices

  dry spices are powdered versions of plant anatomy (flower, seed, bark, leaf, 
  root) which lend an added culinary, medicinal, or \tin{aesthetic} dimension to your 
  dish. to capture a typical indian flavor you will use (among others) cumin, 
  coriander, turmeric, cayenne pepper, cinnamon, clove, cardamom, white\index{color!white} pepper, 
  black pepper, sesame seed, asafetida, bay leaf, curry (neem) leaf, fenugreek, 
  fenugreek leaf, and mango powder. there are age-old specifications of which 
  combinations can be used with which vegetables, and i don't know \tin{anything} 
  about them. go with what tastes good and what makes your body feel good (from 
  anticipation through digestion) and after much trial and error you'll 
  probably arrive at the same conclusions the Ancients did.

  the key with cooking dry spices is to brown them enough to release their 
  flavors. most spices you buy have traveled magnificent distances and have 
  lost much of their potency along the way. any remaining umph can be released 
  through heat --- mix your spices together, sprinkle the mixture over your 
  becoming-curry after the wet spices have been given a chance to brown, and 
  stir constantly. they will threaten to burn and you WILL NOT BACK DOWN. 
  continue to stir until you can smell every element of your spice mixture 
  (even the salt, dammitt) in every corner of the house. only then may you go 
  on. if you stop earlier you will end up with the grainy powdered taste of 
  weakness instead of the full mystical flavors of The Orient.

  \item add the protagonist, ``she who must be curried''

  the curry is essentially ready at this point, missing only the object of her 
  affection. at this point you can add any cooked bean (see chapter zero), 
  leafy vegetable, cabbage-type vegetable, partially cooked tuber or root 
  vegetable, or hard fruit. curries can be made of \tin{anything} from spinach to 
  mushroom to black-eye peas to apples. underripe mango. jerusalem artichoke. 
  pumpkin. every piece of vegetative matter you find has a perfect curry 
  waiting for it Out There in the noumenal realm --- it is up to you to fish 
  for it in the straits of possibility.
  
  lower the heat and cook the curry with the Object for at least five minutes. 
  the Object needs to heat (and maybe cook), the flavors need to infuse, and 
  you need some time to tidy up the kitchen. if you want a wetter, gravier 
  curry, add a little water, tomato, stock, or oil.

  \item finishing touches

  after the curry has melded, adjust the salt. finishing touches are the spices 
  that either don't need to or actively dislike being cooked: lemon juice, 
  sugar, salt, cilantro. many heavily perfumed curries benefit from the 
  lightness of lemon and cilantro; most gujarati cuisine adds sugar to \tin{balance} 
  the burn of spicy food. mix together your additions just before serving, 
  taste, garnish, and send to the table.
\end{algorithm}

\subsection{a variation on chana masala}

\subsubsection{eggplant curry w/ brown rice}

\begin{algorithm}
  \item saut\'{e} onion on high heat in hot oil with mustard seeds and red\index{color!red} chiles
  \item add garlic
  \item add cumin and coriander
  \item add prepared eggplant
  \item add diced tomato
  \item garnish with cilantro
\end{algorithm}

the variation inherent in this dish arises not out of its ingredients but its 
textures. which is to say, before recommending you start substituting celery 
for cumin and cucumber for tomato, there is a lot to be explored in how much 
you cook the onions and in what manner you prepare the eggplant.

to prepare the eggplant, salt circular cross-slices and lay them on a tray. 
yes, i know, you can do it in a bowl but must make sure that each layer gets 
access to the salt or it's all very well tossed together. the salt will draw 
(bitter) liquids out of the vegetable flesh, which must then be rinsed clean of 
impurities.

after perspiration and rinsing, bake the eggplant on an oiled pan. you can bake 
them until cooked through but still holding form, after which it's easy to cut 
circular slices into bite-sized sticks or chunks. or you can cook them longer 
to be mushier (making more of a uniform curry) or you can cook them shorter on 
a higher heat (closer to a broil) to end up with a more assertive eggplant 
identity in your dish.

similarly, you can caramelize the onions slowly on low heat, giving a sweet 
softness and soft sweetness to the dish. or you can fry the hell out of them 
with high heat and little oil, even blackening some edges and adding a mix of 
bitter crispness.

out of what has now been demonstrated to be a VIRTUALLY LIMITLESS field of 
possibility, i usually decided (was party to a decision?) on one of the 
following:

dry and spicy: this version has extra hot chiles, less oil in the beginning, 
sternly fried onions, lots of garlic, much more cumin than coriander, firm 
eggplant, less tomato. it gets a heavier cilantro garnish, an added dose of oil 
at the end, and not much in the way of a soupy-curry-tomato vibe.

sweet and easy: a mellow version with onions saut\'{e}ed on medium-low heat 
just past translucency, moderate garlic and much more coriander than cumin, 
softer small chunks of eggplant (but still not mushy), liberal tomato influence 
and finely chopped cilantro mainly for visual interest.

of course, as always in life love and kitchen it is your sacred duty to TRUST 
and to FOLLOW whatever happens. if She calls while the eggplant is in the oven 
and you're over THERE anyhow trying to roll out tortillas with a soiled bottle 
of beaujolais nouveau, then you're going to have a smoother (not mushier but 
smoother) eggplant curry experience and you are going to love it.

maybe there's still some analysis remaining regarding the destiny of the future 
but the destiny of the past has long been clear to this side of the woodstove 
--- everything\index{everything} She says and does to you happens for a reason, a perfect reason, 
and it's your sacred dharma to find out and to live whatever it may have been. 
we are given the spices and textures to create our own perfection if we can 
just muster the gumption to roll over and Do It.

at every stage of the dish, everything\index{everything} is possible. this is an especially 
beloved dish, probably because most people hate eggplant until they try it, and 
there is no zeal as fervent as a born-again berinjelador. there are only two 
ways to fuck it up:

\begin{ingredients}
  \item too much salt: if you didn't rinse the sweaty slices AND you added salt 
  later on
	
  \item desire is the root of all suffering: if you cook the onions or eggplant 
  differently than some idea you had in your mind, and then can't accept the 
  consequences. the consequences are you altering the rest of your plan (for 
  the dish, for the rest of your life) to take into account your new and 
  wonderful reality\index{reality}, already in progress. if you hold on and fail to do so, 
  that blockage, that resistance, that dissatisfaction will manifest itself on 
  the roof of your friends' and lovers'\index{love!lover} mouths.
\end{ingredients}

\section{chapattis}

chapatti is the standard hindi name for a north-indian flatbread. there are as 
many types of indian flatbreads as there are western risen breads, in a wide 
variety of shapes (circular, oval, triangular), sizes (thin and soft, thin and 
crispy, thick and soft, thick and heavy), and made from different grains 
(wheat, rye, millet, sorghum, lentil, rice, chickpea, and every combination 
implied).

to go into these variations is, as i have oft wanted to say, beyond the scope 
of this book. the simplest technique, to make the oldest bread product we've 
likely known, is as follows:

\begin{ingredients}
  \item flour
  \item water
  \item salt
  \item oil
\end{ingredients}

mix your flour with a little salt and a little oil.

you can choose to add any (or no) spices as well to give your bread a little 
flavor on its own. most flatbreads are meant to accompany a yogurt or curry 
dish, and consequently don't spend much time developing an independent 
identity. traditional basic chapattis use ``atta flour'' which is either a 
mixture of wheat and white\index{color!white} or a milling specification which produces a flour 
somewhere between what we know as ``whole wheat'' and ``white\index{color!white}''. it is fine and 
unbleached. the small amount of oil mixed into the dough provides some 
pliability and mainly serves to keep each bread from sticking to the griddle 
with no added oil necessary.

take (slightly warm) water and mix it over and into the flour.

add water little by little, eventually forming a slightly sticky ball of dough. 
knead lightly for a few minutes but nothing considerable. the dough needs to 
rest for a few minutes and will soak up more water as it does so, so keep your 
ball slightly on the damp side.

when you have your rested ball, you should be able to push and to pull it, to 
tear off chunks and easily roll them into little balls of their own. rip off a 
couple for practice and place these little balls next to your mother lode. 
prepare your tools:

\begin{ingredients}
  \item a large flat area for rolling
  \item a bowl or shallow plate with flour
  \item a rolling pin or clean wine bottle
  \item a griddle or frypan on medium-high heat on the stove
  \item a plate to receive the finish product
\end{ingredients}

you're going to spend exactly as much time rolling a little ball into a circle 
as it takes to cook a circle into a chapatti. in this manner as soon as one 
chapatti finishes cooking, you will be ready to place the next one on the 
grill. clearly you will experience much and terrible failure before you get the 
system down, but it's good to know: the first step depends upon the last.

to roll a ball into a circle, first flatten the ball into a disc between your 
palms. make sure everything\index{everything} is dusted with flour, including the wine bottle. 
roll out from the center of the disc away from you, and then switch directions, 
rolling towards yourself, passing the original center and flattening the side 
closer to you. as you continue the forward/backward motion, think of the wine 
bottle as a steering wheel of sorts, that is leaning right as it goes away from 
you and left when it comes back (or vise versa). in this way you are actually 
rolling a complete circle with each front-back motion. if you are using a 
moderate amount of pressure and everything\index{everything} is firm and well-floured, your 
circular movements should glide the chapatti itself into AUTOMATICALLY TURNING 
as you roll it, creating a perfect circle.

be patient. you may never see the famed AUTOMATIC TURNING. your friend may 
happen by and accomplish the spectral feat on his first try. do not take it 
personally nor as a judgment of your self-worth. we are all equal in the eyes 
of the stomach and every chapatti, no matter how misshapen or mangled, serves 
to pick up another bite of curry. it shall be in the end as it was in the 
beginning and there is only One\index{One!Love} Love\index{love!One Love}.

to cook, toss the thin rolled circle onto the hot pan. make sure there's good 
light and watch the dough carefully. it will start to darken and to change its 
color\index{color}. when the color\index{color} has fully changed (maybe a minute) the underside has been 
cooked and the chapatti can be flipped. you will be able to tell which sections 
were thicker and thinner --- the thin ones may be black or burned while the 
thick ones will have a paler raw color\index{color}. as you roll better, the chapatti will 
cook evenly with no blowouts and no swamps. flip the flatbread and cook for 
another 20 seconds.

if you have a gas stove or an open fire, throw the chapatti directly over the 
open flame for a few seconds and watch it MAGICALLY INFLATE. it's a miracle and 
nobody will ever tell you how it works so don't bother asking. but it only 
happens if your karma is good and you were a focused chef. if not, it's still 
going to taste wonderful, you will make many friends and perhaps even influence 
some of them. try again in the dark hours of the night after sabotaging your 
smoke detectors and listening to bollywood soundtracks. all knowledge is 
memory, waiting hungover within you.


\section{a brasilian\index{brasil} cilantro peanut chutney}

a big hit amongst humans, my mom taught me the antecedent to this recipe after 
much hesitation. it's a jedi\index{jedi} mind fuck to equate its ease of preparation with 
the amazing swirl and fusion of flavor. take the following ingredients, roughly 
chop them until your blending agent can handle it (a bona fide food processor 
won't need any help), and put them in decreasing order of hardness. so, peanuts 
first.

\begin{ingredients}
  \item 200 g \useGlosentry{amendoim}{amendoim torrado}
  \item 2 bunches of cilantro
  \item \onehalf head \gls{alho}
  \item 3 \useGlosentry{limao}{limoes}
  \item 10 \gls{pimentas malaguetas}
\end{ingredients}

the peanuts should be roasted, the lower cilantro stems discarded, the garlic 
peeled, the limes juiced, and the chiles stemmed. at some point the blender 
might stop for lack of water. add coconut milk (watered down if the thought of 
\tin{XXXX} grams of fat / mL doesn't appeal) until it blends well. salt and serve.

this is a very brasilian\index{brasil} adaptation of an indian chutney that calls for twice 
as many fresh herbs and spices. you can pretty much add whatever you want, but 
the peanut / cilantro base is what unites us all.

\subsection{the \textit{pattern} drowning in the green stuff}

the green stuff is always a big hit. the basic premise is that fresh herby 
leaves should be packed together for powerful\index{powerful} flavor and possible preservation, 
though unless you seal a jar off, people will devour it all immediately. the 
green is often blended with strong spices like garlic or ginger and a source of 
fat and protein (some sort of nut).

for some plants, such as parsley, cilantro, arugula, or spinach, preparing the 
leaves is easy. you chop off most of the stemmy section and blend the rest in 
with the leaves without much worry. woodier herbs such as basil or mint offer 
the opportunity for more work and dedication --- the best way i've found is to 
hold the branch by the head with one hand and run the other hand down the 
length of the stem, ripping off all leaves in your path. you're left with a 
naked stem and a crown of leaves which you can pop off in a simple motion and 
pick up the next branch.

nuts for green stuff recipes are generally toasted to bring out a deeper 
flavor, but can be raw as well (as in the typical indian cilantro-peanut 
affair). if you plan to make a lot, you'll be peeling a lot of ginger or 
garlic, and it helps to know the tricks\index{tricks}:

\begin{ingredients}
  \item for garlic: to peel quickly, chop off the stem-side tip and flatten 
  harshly with the side of a chef's knife (or a coffee cup). the skin should 
  fall effortlessly off the crushed clove. if it doesn't, swear gently and 
  crush again.

  \item for ginger: to minimize flesh loss while maximizing peeling speed, use 
  the back of a spoon. it has just the right dullness to remove the peel 
  without getting caught up in the meat.
\end{ingredients}

robots\index{robot} are the preferred way of quickly preparing the green stuff. you can have 
a course out (pesto pasta) in as much time as it takes water to boil with a 
\tin{robot} and a little experience.

\subsection{variations on the green stuff}

\subsubsection{pesto}

\begin{ingredients}
  \item basil
  \item garlic
  \item pinenuts
  \item hard cheese
  \item salt
  \item pepper
\end{ingredients}

\tin{robot} together, using olive oil liberally as a lubricant. i often use walnuts 
instead of pinenuts when i can find them on the forest floor, and tofu for 
parmigiano when cooking with vegan humans in mind. using a large amount of 
garlic and roasting it in olive oil ahead of time gives a deeper and suaver 
dimension to the pesto's bite.

\subsubsection{salad green pesto}

if you're lucky enough to get spinach, mizuna, arugula, or a similar green, and 
can't make salad out of all of it, throw them together for a great pesto. since 
they don't have as much history or fragrance\index{fragrance} as basil you'll need to have more 
spices or lower expectations of the flavor power.

\begin{ingredients}
  \item even mixture of mizuna and arugula
  \item even mixture of garlic and ginger
  \item as many walnuts as garlic and ginger walnuts
  \item half as much feta cheese as walnuts
  \item a little soy sauce
  \item a small green chile
\end{ingredients}

\subsubsection{cilantro peanut chutney with pineapple}

another brasilian\index{brasil} variation on the cilantro peanut chutney (the green sauce) 
popular with samosas; the tartness and fluidity here comes from fresh pineapple 
juice.

fresh pineapple juice is simply a pineapple in the blender and then through a 
strainer to separate the foam and any eyes you didn't properly exorcise.

i blend together the following:

\begin{ingredients}
  \item 1 bunch of cilantro
  \item 1 bunch of mint (just the leaves not the stems) 
  \item \onehalf \useGlosentry{alho}{cabe\c{c}a de alho}
  \item \gls{gengibre}
  \item a handful of small hot chiles
  \item 200 g \useGlosentry{amendoim}{amendoim crudo}
\end{ingredients}

start with the peanuts to get a sort of crumbly peanut butter, add the sharp 
spices (garlic, ginger, chiles), and then the herbs. when it stops blending 
smoothly add \onequarter of the pineapple juice and continue adding pineapple 
juice until the chutney is the right texture. you will have a lot of juice 
leftover and may want to convert some of your chutney into a marinade or 
dressing by thinning it further. or you could use the pineapple juice as the 
base for a new salad dressing. it will be so fresh and good that cooking it 
would be a goddamn\index{god!damn} shame and maybe you should just think no further and drink 
it. that's what we always did, anyhow.

\section{gujarati carrot salad}

i'm not actually sure if this is something i've ever eaten in somebody else's 
kitchen, but the pattern works in general. indian ``salads'' are often just one 
or two vegetables, grated or finely chopped, with a piquant oily dressing.

\begin{algorithm}
  \item grate a bunch of carrots to an appropriately digestible size.

  \item heat canola oil (mustard oil is better) on the stove, and when loose 
  and easy, add turmeric and black (or yellow\index{color!yellow}) mustard seeds.

  \item you can add some grated ginger as well, if you're feeling in the mood.
\end{algorithm}

soon the seeds will pop in the oil: cover or end up with some combination of 
sculpture and burns. when they finish popping, pour the oil and seed mixture 
over your carrots and mix well.

the salad is basically done --- i sometimes add vinegar to help the carrots 
marinate. it's much better after sitting a few hours than right away.

\subsection{a \textit{pattern} for gujarati carrot salad}

the idea here is just to make a spiced marinade on the stove and mix with a raw 
vegetable. you can do the above dressing (hot oil and mustard seeds) with 
cucumbers, onions, carrots, cabbage, or \tin{anything} else you want. toasting 
coriander and cumin seeds provide another base. variations might include upping 
the amount of vinegar and mixing in a little sugar, taking the dressing in 
vietnamese direction.

\subsection{variations on gujarati carrot salad}

\subsubsection{towards a carrot pickle}

a variant of the carrot salad which people absolutely love is to throw the 
carrots into the oil instead of the oil into the carrots. this requires more 
oil, of course, which is why people love it. you can then add more spices, 
vinegar, and cook it all together until the carrots become tender. when you 
serve it change the label from salad to chutney and people will be psyched. 
however, you lose the raw fresh goodness of the vegetable. if you're concerned 
about that sort of thing.

\begin{ingredients}
  \item a lot of carrots, diced neatly
  \item equal and large quantities of oil and vinegar (enough to cover the carrots)
  \item ginger and garlic
  \item turmeric and coriander
\end{ingredients}

heat the oil in a pan and add the spices (ginger, garlic, turmeric, coriander, 
and salt) when hot. let sizzle and cook until the ginger and garlic change 
color\index{color}. add the carrots and vinegar and cook together until the carrots are 
tender. cool and baptize ``quick carrot chutney''. you can bottle and store or 
keep in the fridge --- the oil off of the top will be great for salads and 
marinades.
