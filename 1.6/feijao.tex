\mychapter
{chapter_corn.jpg}
{feij\~{a}o . arroz . mango salsa . picante}
{refried feij\~{a}o vermelho e arroz vermelho accompanied willfully by
mango salsa and the standard brasilian table hot}

\label{chap:feijao}

\section{red rice}

according to the kama sutra, if i recall, there are sixty-four tasks
or talents a courtesan must be able to perform with superior skill,
and \textit{one} of them is to prepare rice in 40 different ways. the
most simple -- no matter what your color or size of rice is --- is
boiling.

\begin{ingredients}
  \item 1 cup of rice
  \item 2 cups of water
\end{ingredients}

the longer and browner (the more hull each grain has, the more mass,
the more density) your rice, the more water you will need. really
short white rounded grains (which you shouldn't really be using if you
have a choice) will need less. so a really long grain brown rice could
take 2\onehalf cups of water, and sushi rice takes just 1\onehalf.

IT DOESN'T REALLY MATTER THOUGH because this method replaces the care
of measuring with the care of attention. once you've figured it out
for your particular rice you can resume the cruise control you've been
c(l/r)utching this whole damn time.

if the rice looks like it might have some teeth-breaking rocks in it,
like the organic red rice we bought in brasil, sift through it in a
pizzapan to remove any offenders. if it looks web-y, dirt-y, or
otherwise like it came directly from the earth, rinse it.

place it with the water to boil on high heat. i add a little bit of
fat (oil or butter) and salt. this serves to keep the rice from
sticking and to bring out a bit of flavor, but mainly it's a
ritual. you can add anything you want --- cloves, soy sauce, bacon
bits, whatever.

when the water boils, reduce the heat to a simmer and cover. the rice
will steam and absorb the water. if you have done the proportions
perfectly, after a certain mystical amount of time

\begin{enumerate}
  \item[a)] all the water will be gone 

  \item[b)] the rice will be perfectly cooked
\end{enumerate}

most likely this will not be the case. when you check (and don't
wait too long because you can't unburn it), you will compare the
amount of water left with the amount the rice needs to cook. if
it's basically done and there's still water swishing around, all
you have to do is pour the water out of the pan and set it back to
keep cooking. if the water's almost gone and the rice is still raw,
all you have to do is add some water. it helps to add the water in
small batches so it can be heated without lowering the temperature of
the rice too much.

since you've already ruined the steampressure situation by checking
the rice, take a few seconds to stir it against burning or
sticking. you'll get a good sense of its moisture as you do this.

know that even after you turn the heat off the rice will continue to
cook and absorb a little more water, so if you check it a little
later, it's already done and there's a bunch of water, EVERYTHING WILL
BE ALL RIGHT. drain the rice well (use a strainer if you have to) and
then set it back in the pan with the lid on to finish. don't check it
again until you eat to demonstrate your faith.

\subsection{the \textit{pattern} underneath arroz vermelho
\footnote{arroz vermelho (br) : red rice}}

\begin{itemize}

  \item the basic outline for boiling rice is:

  \begin{algorithm}
    \item heat an empty pan
    \item add some sort of lubricant
    \item add the rice
    \item add the liquid (usually 2:1) with salt 
    \item turn down the heat when the water is boiling; cover
    \item let steam to perfection, turn off, fluff.
  \end{algorithm}

  \item in a standard boiled rice you don't necessarily have to add
  butter or oil (step 2) though it does help with sticking, especially
  for white rice (more glutinous).

  \item in between many of these steps you can add spices or other
  vegetables to cook with your rice. they should be added according to
  hardness --- potatoes at the beginning, greens at the end. spices
  which require browning can be added between steps 2-3 to roast and
  release their flavors into the oil.

  \item remember that you can mix different types of rice together, or
  even rice with other grains (see chapter six). also, keep your mind
  open to the notion that liquid is not necessarily synonymous with
  water: you can use old soup, vegetable stock, the cooking water from
  beets or potatoes, some milk, coconut milk, tea, coffee etc. alcohol
  is also popular (like in risotto), and while i've never found much
  of a taste difference, it looks great on a menu (... accompanied by
  a georgian wild rice earnestly simmered in a white wine reduction
  ...)

  \item soaking the rice for a few hours (or overnight) will greatly
  reduce the cooking time, because the rice will already have absorbed
  a significant amount of water. it's another way that a little bit of
  work (perhaps 30 seconds) the night before can save you ten minutes
  of cooking while hungry the next day. this is a cornerstone of my
  argument against the supposed economy of ``fast food'' --- the food
  is not actually cooked or prepared any faster; \textit{you}, as a
  narrow alienated individual, have to deal with it less. to compete
  with that mentality (and compete we should, because who wants to
  wait an hour for dinner after spending all day working for the man),
  you must learn a handful of slight and clever tricks to save
  yourself that precious illusion they call time. all this and more in
  pamphlet ``how and why to fuck the system by cooking your own food
  (and your neighbors' goddammit)'', coming out later on in the
  kaliyuga.

\end{itemize}

\subsection{variations on red rice}

\subsubsection{clove or cinnamon pul\~{a}o}

the first and simplest rice i learned how to make was an indian
pul\~{a}o, which follows the technique outlined above with these
variations ---
\begin{itemize}
  \item use long grain basmati rice if you can get it

  \item the rice is generally soaked and rinsed ahead of time

  \item one uses ghee\footnote{ghee (hindi) : clarified butter} in
  step 2
  
  \item you throw in a few whole black peppercorns and whole cloves
  with the rice and water
\end{itemize}

this can also be made with brown basmati. it has the perfumed aroma of the 
basmati with added sweet accents of black pepper and clove.

cinnamon perfumed rice is made in a similar way, except that i'll
saut\'e a cinnamon stick (broken into two or three removable pieces)
in the butter for a few minutes before adding the rice and water. for
an added kick you can add a \onehalf teaspoon of ground cinnamon to the
water as well.

my mom will typically stir in a little butter at the end and if you're
ever trying to impress some non-vegans, you should follow her lead.

\subsubsection{magic yellow rice}

this simplest variation is merely to add \onehalf teaspoon of turmeric to
the water when cooking any white rice. turmeric is a powerful spice,
medicine, and dye, and even a tiny amount will dye the entire pot of
rice yellow, a boon for those mindful of plating aesthetics.

\subsubsection{magic purple rice}

another simple variation puts one diced beet into the water with the
rice. the cooking beet will turn the entire pot of rice purple, which
most people have never even imagined, must less witnessed, in their
mortal histories. the sweet beet flavor goes well with the
clove/pepper pul\~{a}o variation above.

\subsubsection{sweet potato biriyani}

sweet potatoes were always in the market, though never very good. but
sometimes you get tired enough to use them. this recipe works equally
well with yams, and likely tastes much better.

\begin{ingredients}
  \item 1 large starchy tuber
  \item 1 carrot
  \item 2 cups of rice
  \item 4 cups of water
  \item 3 bay leaves
  \item 2 tablespoons of raw peanuts
  \item cinnamon
  \item clove
  \item black pepper
  \item butter or oil
\end{ingredients}

prepare to boil the rice as suggested above. grind together the spices
(or get them ground if you don't have the means of production to do it
at home) and saut\'{e} them for a minute in the butter. add the rice
and stir well with the butter and spices, allowing it to toast lightly
before adding the liquid. with the liquid add the sweet potato,
chopped into bite-sized chunks, as well as the bay leaves and
peanuts. when the rice reaches a boil add the carrot in small diced
cubes, simmer down, and cover.

\subsubsection{fatty coconut rice}

replace half of your cooking water with coconut milk, stir well in the
beginning to avoid scorching, and cook normally. adding a bit of mango
pur\'{e}e or diced mango will give added sweetness and bring further
to mind the Absolute Perfection of the tropical world. a bit of
toasted coconut or almond makes an excellent garnish.

for a fatty rice experience without the coconut pricetag, try using
milk, half and half, evaporated milk, or condensed milk for some
fraction of the cooking liquid.

\subsubsection{spanish-style fried rice}

there are two types of rice commonly called fried rice. the first one
rolls the rice around, still dry, with spices and onions to the point
of browning, then dramatically takes down the temperature using the
cooking liquid.

\begin{ingredients}
  \item 1 diced onion
  \item some celery (optional)
  \item a mix of cayenne pepper, cumin, paprika, and oregano
  \item 2 cups of rice
  \item 4 cups of water
  \item salt
\end{ingredients}

use something closer to a frypan than a pot for this variation. start
by heating one tablespoon of oil in the pan and add the rice when
hot. continue to flip and stir the rice, coating it with the oil. when
it starts to look a little drier, add the spices and stir together. as
the temperature will be hot, they should release their flavors and
start to perfume quickly. when the rice begins to brown, add your
onion and celery. this will take down the heat a little bit, buying
more time for the onions to cook and the spices to develop before the
rice wants to burn. when the rice starts to brown yet again, add the
cooking liquid. using vegetable stock or chopped tomatoes (maybe from
a can, even!) makes an appreciable difference. simmer when it boils
and stir occasionally --- the tomatoes and red spices should give the
familiar orange color.

the cheesily minded should not pass up the opportunity to sprinkle the
rice with cheese when it's almost done, and either cover or broil to
melt it.

\subsubsection{chinese-minded fried rice}

this recipe follows the second type of fried rice i have
encountered. it's a recycling process of yesterday's dinner into
today's lunch, and the only chinese connection i can imagine is
that it works best with soy sauce.

\begin{ingredients}
  \item leftover rice
  \item some cooking oil
  \item onions
  \item garlic
  \item soy sauce
  \item sesame seeds
  \item perhaps an egg
\end{ingredients}

heat oil in a pan and chop your onions and garlic. fry them on medium
heat until translucent. while they are frying add a little water and
soy sauce to the rice container and break up the chunks with your
fingers. when the onions are translucent add the rice and turn up the
heat, stirring vigorously to separate the rice and mix well with your
spices. if you eat eggs, break in an egg or two at the end, throw in a
couple shakes of sesame seeds, and continue cooking (back to medium
heat) until done.

this is an excellent rice to accept chopped greens (spinach, beet
greens, chard) as well, just before you would add the egg.

fluff and garnish with finely chopped green onions before serving.

\section{refried red beans}

ajwain/ajmoda is an indian herb that i've seen explained as oregano
seed, thyme seed, and celery seed. it's likely something else
entirely. i have found that it goes quite well with the excruciatingly
popular cooking oil of bahia, dend\^{e}. azeite-de-dend\^{e} is made
from little black olive looking berries clustered around the trunk of
a palm tree. it has a reputation for being fatty and unhealthy, which
translates to a spicy heavy flavor, a strong aromatic presence, and a
deep red color.

\begin{ingredients}
  \item some tablespoons azeite-de-dend\^{e}
  \item some tablespoons ajwain seeds
  \item one medium onion, chopped, for each cup of cooked beans
  \item many cups of cooked red beans (red pinto looking creatures)
  \item some ground cumin (optional)
\end{ingredients}

pcook the red beans (see explanation below) and drain them. this would
of course work with any bean but the whole red thing wouldn't fly as
smoothly. heat a saucepan that can take some punishment, to the high
side of the fire. when it's hot add enough dend\^{e} oil such that it
lightly covers the bottom of the pan.

toss in the seeds and let them sizzle and pop. their flavor will fill
your kitchen and when it does (but before they blacken) add the
chopped onions. fry them vigorously and attentively on high heat until
they brown. if they start to stick or otherwise don't seem right
before they get to be browning, add some more dend\^{e}. when the
onions are literally brown you can add the cumin. doing so will make
the flavor more familiar and likable --- the ajwain can be strong and
foreign on its own. remember to brown the cumin, cooking it on high
heat until you are shocked by the pungency of its aroma, before you
add the beans.

now add even more dend\^{e} with the beans and stir everything
together. reduce the heat to medium and spend the next few moments of
your life BEING the mixing and crushing of those beans. no bean left
intact: some a paste and most mere fractions of their harvested
selves. this is the well-frying of the beans, where they come face to
face with the vermilion specter of dend\^{e}, absorb the depth and
pungency of the spices, and evolve into the alreadiness they have been
awaiting.

salt to taste, serve.

\subsection{the \textit{pattern} of refrying beans}

the first thing the people need to know, before they take either
control, the streets, or the power, is how to cook beans. somewhere in
this book there is a section on Equipment and therein lies the answer.

once you have quickly- and safely-cooked beans, and forever evicted
the twin scourges of tin cans and excessive flatulence from your
hallowed kitchen, you can start getting into the wonderful world of
leguminous protein in all its protean variations.

the frijol refrito\footnote{frijol refrito (es) : commonly translated
as ``refried bean''. accurately translated as ``well-fried bean''} is
probably (and justifiably) the most famous of these. a typical
refried-bean pattern (USE THIS FOR ANY and EVERY DRIED BEAN) goes
something like this

\begin{algorithm}
  \item pcook your beans
  \item while they are cooking
  \begin{algorithm}
    \item chop some sort of onion
    \item prepare (usually roast and grind) some sort of spice
    \item heat oil in hot pan (medium-high)
    \item fry onions until browned
    \item release pressure from cooker (it should be done)
    \item fry spices until smelly
    \item add beans, saving water off to the left
    \item cook together with low heat and high energy, mashing until
    it looks like it came out of a cheap can
    \item add last minute touches and salt.  
  \end{algorithm}
\end{algorithm}

note that:
\begin{itemize}
  \item indians put a small amount of asafoetida (which tastes
  horrible) with all beans as a digestive aid.
	
  \item most beans go well with garlic, which should be added as the
  onions show evidence of browning.

  \item each bean goes well with some spices/herbs and not others. let 
  reckless experimentation be your guiding star.

  \item final elements often include cilantro, lemon juice, salt, and
  sugar; important additions for the palate that neither want nor need
  to be heated.
\end{itemize}

\subsection{variations on refried beans}

\subsubsection{refried feij\~{a}o fradinho dip
\footnote{feij\~{a}o fradinho (br) : black-eyed peas}}

use the pattern above.

so for the feij\~{a}o fradinho that are so ubiquitous in bahia, i
applied this technique to green onions (excising the root end and the
sad looking sections of the top), fenugreek seeds and garlic. to be
precise, i threw the garlic in between steps (d) and (f) while i
performed (e) with my other hand. it's an advanced maneuver.

fenugreek is an indian plant and you'll have to BUY the hard angular
seeds from an ETHNIC store. they are too hard to use raw and turn soft
and pleasant (and bitterly toned) when dry roasted, allowing easy
grinding. if the last sentence really turned you off, you can use
cumin but cumin is such a staple spice with this sort of dish so
please for the love of martha try something else. i imagine dill- and
pumpkin-seed would both give interesting flavor experiences.

for the final touches, chopped cilantro and lime juice are essential
as few other things in this world.

a note on nomenclature: this is a dip, not a plate of beans. why?
essentially it comes down to the power structure appropriating
language to manipulate those among us who recognize it only on the
level of a communication tool and neglect to notice the deep
mindwarp-level where it really simmers.

specifically, you've mashed the beans enough to kill any agency or
identity in- and for- themselves, so they just default to a mashy
hive-mind we call `dip'. the consistency is thick and starchy enough
to balance heftily on a bread or chip, and the spicing a bit too
prominent to be a main dish anywhere outside of the indian
hypercontinent.

\subsubsection{standard and delicious black beans}

use plenty of garlic when the onions are browning and add the
following spices when the garlic browns:

\begin{ingredients}
  \item lots of ground cumin
  \item oregano
  \item chopped rosemary (optional)
\end{ingredients}

brown well until mighty and fragrant. when you add the beans use some
of the reserve liquid (and perhaps a chopped tomato) as well to
provide a lounge for a couple of bay leaves. simmer down the water for
ten minutes, allow the bay leaves to express themselves before removal
and mashing begins in earnest.

these are excellent with finely chopped onions and cilantro mixed in
at the end, perhaps with a little lemon or lime juice.

\subsubsection{feij\~{a}o fradinho \'{a} la mostaza
\footnote{mostaza (es) : mustard}}

black eyed peas are all over bahia (famously in acaraj\'{e}) and
frequently showed up at our table.

before adding the onions to your hot oil inject a heaping tablespoon
of whole black or yellow mustard seeds. cover the lid in anticipation
of savage popping and add the onions only when the music ceases. fry
the onions closer to medium than high heat and add the garlic when
translucent. add chopped tomato along with the beans, simmer in a
little of the cooking liquid (as in the black beans, above), and stir
more vigorously when the liquid vai embora.\footnote{vai embora (br) :
goes away}

we never had any in bahia but chopped mustard or collard greens would
fit perfectly in this dish, in lieu of adding extra cooking liquid at
the end.

\subsubsection{feij\~{a}o fradinho with fenugreek and urucum}

another black eyed pea dish, this one comes out spicier and drier.

after the onions are brown add chopped garlic and green chile, brown
for a minute, and add a toasted ground mixture of fenugreek and
urucum. urucum, if you have it, is a red seed used mainly as a
colorant in bahia. fenugreek is more commonly used in indian cooking
and has an earthy, slightly bitter flavor. both are extremely hard and
require soaking or toasting to be able to grind them properly.

continue as in the standard recipe, adding in farinha de mandioca or
cornmeal in the end to cook for a few minutes and give a rougher
texture.

\subsubsection{green pinto bean tacos}

i haven't seen them before or since, but green pinto beans refried
with cumin and coriander made an excellent taco lunch one hot december
day. i'm sure normal pinto beans would do the trick.

a slight unspoken glass of coconut milk gives the beans an added kick
of creamy consistency.

serve with freshly ground roasted cumin on top.

\section{mango salsa}

\begin{ingredients}
  \item 3 ripe mangos
  \item 1 good white onion
  \item 1 bunch cilantro
  \item 6 malagueta chiles
  \item juice of 3 limes
  \item salt
\end{ingredients}

the perfect salsa for a tropical climate, founded on the early 21st
century mantra --- ``the mango is the new tomato''. of course, in the
old tropical world, the tomato was once the new mango. but never as
good.

chop everything together. ingredients are listed in decreasing order
of size. the bigger and sweeter your mangos are, the smaller and
hotter your chiles should be. remember that lime juice and salt are
your bridges and if something tastes wrong, you needs must build more
bridges.

\subsection{fruit salsa as a \textit{pattern}}

here the world is your oyster and no direction is advised. a good
fruit salsa by my standards is sweet, hot, and sour. so take any
fleshy fruit from peaches to papayas, add lime, onion, and chile, and
you're on the road to success. some fruits go well with ginger, others
with garlic, and many with nothing but pure hot chile peppers.

to experiment, walk around the grocery store or market with a chile in
one hand and rub it over every fruit you see before taking a big
slobbering bite. by the time you're evicted you should have a good
idea of how to proceed.

\subsection{variations on mango salsa}

\subsubsection{pineapple salsa}

\begin{ingredients}
  \item 1 pineapple peeled and chopped
  \item 1 bunch of cilantro, stemmed and chopped
  \item 1 medium white onion, diced
  \item a small amount of ginger and chile pounded together
\end{ingredients}

\subsubsection{pineapple salad}

pineapple is so damn good you don't actually need anything else. cube
pineapple and mix it with salt and black pepper for a savory sweet
spicy dinner salad salsa.

\subsubsection{soy peach / papaya salsa}

this salsa worked really well as a marinade one night when Wagner
surprised us with an eel.

\begin{ingredients}
  \item 1 ripe papaya
  \item an inch of good ginger
  \item juice of two or three limes
  \item 1 tablespoon of honey
  \item soy sauce to taste
  \item cheap white wine to thin it out
\end{ingredients}

the papaya should be ripe enough that cubing it leads to a small
mess. mince the ginger, mix all the spices together well, and add the
papaya gently so as not to pur\'{e}e it.

if you're from nowhere near the tropics, it's still important to look
at this and make it your own. really ripe peaches are papayas in
disguise and this recipe would be perfect for them, as well as
overripe or poached asian pears and especially those overripe
persimmon creatures (kaki in various languages) you can get on the
cheap during the dying hours of the public market in the cours
salaya.\footnote{cours salaya (fr) : the market street in old nice,
france}

\section{standard brasilian table hot}

so yes in this particular case `hot' is a noun. you can (or at least,
i could) go into any restaurant in bahia and ask ``tem picante'' (do
you have hot?) to which the baiana would respond ``teeemmmmm''
(haaaaaaave) or maybe just bring you a little bowl of spicy vinegary
salsa. depending on where you are eating you taste more chile or more
vinegar. no matter where you are, asking for salsa is no good ---
they'll just bring you parsley.

\begin{ingredients}
  \item a cup of malagueta chile peppers
\end{ingredients}

or any of their small spicy brethren. they're cheapest at the end of a
market (get the ones that are going bad and pick through them, singing
along to gilberto gil) or at a funny eastern grocery anytime.

\begin{ingredients}
  \item some under-ripe tomato
\end{ingredients}

what we/they have in brasil; your standard supermarket
hard-and-flavorless should do the trick

\begin{ingredients}
  \item an onion
  \item a few cloves of garlic
\end{ingredients}

chop everything together into a small heterogeneous mixture. the
peppers should be the smallest pieces and the largest total
volume. add a little salt and pepper and cover with vinegar. let sit
for a day or two, occasionally shaking. the infusion will of course
get stronger and stronger as the days go on.

to use, you can either pour off the vinegar directly for salad
dressings and medicinal purposes, or spoon out the mixture with its
juices, like in brasil. always cover with vinegar after using to
prevent the quick spoilage that is the specter haunting tropical
lands.

\subsection{the picante \textit{pattern}}

the simple idea here is that the material world is none other than
your personal chalkboard, and you should look at gallon jugs of
distilled vinegar in the same way a sculptor looks at driftwood or a
child at construction paper. anything that tastes good by itself ---
and many things that don't, due to strength or edginess --- will make
vinegar taste great. and there's no reason to use flavorless
industrial vinegar ever again, not when you have rosemary, garlic,
dill, ginger, cucumber, or chile to spare.

the next time you're leaving the market with a measly two points in
your pocket, you may happen to pass a tired farmer trying to reli(e)ve
himself of a produce basket whose memebers have lost their virginal
luster.

\hskip2cm fear not. hesitate not.

clear him out, wash them well, add salt and pepper, and cover with
vinegar for a whimsical while. you'll have pickled XXXX and tasty
vinegar in a few weeks (or months) to validate your most minor of
charities.
