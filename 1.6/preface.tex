\mypreface{preface.jpg}{Preface}

Prepare to meet a recipe book unlike any other you've ever read.  I
don't mean to embark immediately on flights of hyperbole here in this
preface to the alchemical\index{alchemical} stew that awaits, but in this case, I'm
simply calling it like it is.  Give it a chapter and tell me you're
not a convert.

The \tin{Bigode} restaurant flourished briefly on the Island of Itaparica
off the coast of Salvador in the Brazilian state of Bahia between
September 2004 and February 2005.  As a real honest-to-God\index{god} attempt at
founding a restaurant in which spontaneity and improvisation ruled the
day, the \tin{Bigode} certainly veered between successful orbit and meteoric
collision.  Imagine attempting to found a flowering vegetarian
establishment in the dense overgrowth of the most militantly
carnivorous environs imaginable. (Brings to mind the subtle realm
mantra ``Location, Location, Location...'') On top of its, say it, out
of place vegetarian aspirations, the \tin{Bigode} also attempted to usher in
a new era of idealistic gift-economy relations by posting no prices
for its offerings and accepting donations only in exchange for its
love-crafted foodstuffs.  You can probably fill in the gory details
for yourself.  Each weekend, the \tin{Bigode} opened for business around
noon and offered piping hot vittles straight through to nightfall.
Most weekends, the clientele consisted of one or two friends of the
proprietors and perhaps the occasional hapless local inquiring about
beef stew or the like.  As has so often been the case with
trailblazing artists and explorers, the true genius of the \tin{Bigode} did
not gain wide recognition during its brief passage through space-time\index{space-time}.

With this book, the recipes of the \tin{Bigode}, nearly lost to history's
compost bin, are preserved for posterity.  The dishes you're about to
digest are all, without doubt, miraculous, battle tested and
nutritionally complete.

You'll quickly notice upon perusing the ensuing incantations that the
author of this work places much emphasis on experimentation, variation
and attention to the moment in which the dish in question arises in
universal awareness.  I hope you aren't too troubled by this---for it
is the way of things, always has been, and it's time we accept the
unique demands of the present moment in spite of its seemingly
infinite repetition of the motifs of the past.  Where most recipe
books include helpful quantities like \onequarter tsp or \onehalf cup, in
this manual you'll often encounter the mysterious cypher \tin{XXXX}.  Some
readers might be inclined to feel adrift and listless in the face of
such unspecificity.  Don't!!  There's method to this madness, just as
all the swear words to come are perfectly justified depending on
context and authorial intention.

Also, dogmas (doctrinas in the much preferable Brazilian Portuguese)
won't last long in confrontation with the avant garde recipe-forms to
come.  If you find yourself preferring rigidity to the tensile give of
the finest trees/\-skyscrapers/\-suspension bridges, you'd do well to
trust, let go and let God\index{god}.  You'll be happy you did.

So, jettison your preconceptions and prepare to confront your kitchen
for what may well be the very first time.  While the \tin{Bigode} may never
have located its local following in large enough numbers to support
its long-term plans, we have preserved before us in the form of this
recipe book the beautiful seeds from which all further Bigodes\index{Bigode} shall
spring forth.  With a little close attention, some locally available
produce and the will to adventure, this guidebook will carry you
through to glory.

Also, for those who are interested, \tin{Bigode} means ``moustache.'' As in:
What's a vegetarian moustache like you doing in this bastion of
carniverosity??

Free your minds, intrepid chefs of the New Time.

\bigskip
\bigskip
\bigskip
\bigskip
\bigskip
Matt Coffman \\ Brooklyn, New York \\ February 2006
