\mychapter
{chapter_prelim.jpg}
{some preliminary notes}
{some preliminary notes towards the construction of bigodean\index{Bigode} recipes}

\section{some preliminary notes}

the point of this cookbook, as much as to communicate a sense of the
style and content of O \tin{Bigode}, is to bring the means of Culinary
Revolution to the masses. it comes out of a deep discomfort i have
with the modern food system (and the social system in general): a
feeling of loss and betrayal that we've allowed so ancient and sacred
a ritual to be colonized by people, machines, processes, and
world-views totally alien to us.

as always, you have had the power all along, and i'm just doing what i
can to demonstrate that. as such, i have tried to make the recipes
herein independent of time, place, wealth, or social scene. most
probably don't even require fire, and only seldom do i proclaim an
ingredient to be mandatory.  our foods, in dialogue with our emotions,
change in texture and temperament with the days and seasons --- i see
these recipes as mere pattern to help you organize what comes home
from the market each particular sunny afternoon.
  
\section{equipment}

i do suggest some tools, primarily to help you fit these recipes
(which many might find complicated and challenging) into the rhythms
of your everyday life. canned beans, packaged spices, and prepared
pestos have no place in the world we're creating, and the small
investments you may choose to make will ultimately save you hundreds
of points\index{points} and hours.

\begin{description}

  \item[a knife and knife sharpener] buy a knife that can be
  sharpened. you do not need a fancy or expensive knife, but a knife
  that you can make sharp at will.

  \item[a cutting board] many cultures\index{culture} hold their produce in one hand
  and cut with the other. if you try this you will hurt yourself. get
  or make a large one.

  \item[a strainer] for making tea, juicing lemons, and draining
  beans. not necessary but saves a lot of stress when Everything\index{everything} is
  happening at once.

  \item[a \tin{mortar} and pestle] a heavy, metal mortar and pestle from the
  local indian store. wooden ones get tainted by garlic. a must for
  catalyzing fresh and fragrant spices. generally, you will toast your
  spices (without oil) on high heat until they change color\index{color}, then pour
  into the \tin{mortar} and pestle to pound into dust. the mortar also
  serves to mash together garlic, ginger, onion, and green chiles for
  curries, to crush peanuts or walnuts, and to grind rock salt and
  peppercorns.

  \item[a pressure cooker\index{pcook}] fundamental. the one thing you have to
  buy. people who don't have pressure cookers\index{pcook} waste half of their
  lives waiting for protein or end up building up mini tin reserves
  and probably supporting the war. it might run you fifty dollars but
  goddammit\index{god!damn!it} it's worth it. to use, first soak your beans overnight (or
  four hours) in water. the more often you change the water, the less
  magic bean gas you will experience. drain your soaked beans and
  place in the pressure cooker\index{pcook} with fresh water. cover the beans and
  then some. bring the pot up to pressure (there are different styles)
  and cook until it whistles a few times or otherwise indicates the
  beans are ready. let the pressure cooker\index{pcook} cool gently before
  opening. it's important to understand that the pcooker\index{pcook} is a powerful\index{powerful}
  spirit and should be treated as such. do not misuse or disrespect it
  in any way because IT WILL HURT YOU. honor and respect the pcooker\index{pcook}.

  \item[a \tin{robot}] often called a food processor. also extremely
  helpful. if you have a large (i mean large) \tin{mortar} and pestle that
  can work but the \tin{robot} might just be the third joyous invention of
  the twentieth century (\textit{pace} dr. walter vogt) and i suggest
  you get one. for making hummus, soups, p\^{a}t\'{e}s, and salsas,
  it's truly extraordinary. competing memes might point to a blender
  or a salsa-maker --- the former isn't powerful\index{powerful} or wide enough and
  the latter, while being a narrow manifestation of miracle in its own
  right, only works for (tomato and onion) salsa.

\end{description}

\section{specialty foods}

while, programmatically and philosophically, i generally suggest
making everything\index{everything} from scratch and doing without those ingredients
only available in the dying oil economy, there are a few really
wonderful luxuries worth buying, in cans, from far-away locales. they
are:

\begin{description}

  \item[chipotle peppers in adobo sauce] unless you can get them dry
  for cheaper (i doubt it)

  \item[coconut milk] again, if you don't have coconut palms nearby

  \item[soy or tamari sauce] you can make sure it's 100\% soy or you
  can give in to the reality\index{reality} that in most countries it will be cheap,
  wheat-based, kind of nasty, and still wonderful.

\end{description}

\section{what you will not find in this book and why}

\begin{description}

  \item[desserts] we ate and served fruit.  mangos, papaya,
  passionfruit, coconut, guava, tamarind, jambu, starfruit, etc.

  \item[mushrooms] we couldn't find any except at the bourgie
  markets. and we love them. just love them. like dolphins\index{dolphins}.

  \item[pasta] everybody can make it. and it's not vegan.

  \item[drinks] there were too many to list. they can be found in the
  forthcoming companion volume:\\``My Brother Drinks: O \tin{Bigode} Cocktail
  Hour''\\by kevin panozzo and erikki uzureau

\end{description}

\section{acknowledgements for the 1\ordinal{st} new york edition}

this book is dedicated to our mothers, \textbf{bharti} and
\textbf{madonna}.

madonna is managing the distribution and sales from tennessee. thanks,
madonna. \textbf{george} in nashville stayed up all night
printing. thanks, george. \textbf{erik}, \textbf{max}, and
\textbf{matt} did the editing. erik thought up the glossary. max drew
up the artwork. matt did up Everything\index{everything} Else. thanks, brothers.

either \textbf{\tin{amanda}} or \textbf{cholmes} dreamt the idea for a
\tin{Bigode} cookbook. \tin{amanda} and i worked and played together in a \tin{Bigode}
kitchen. cholmes promised a \tin{Bigode} tatoo. thanks, \tin{amanda}. thanks,
cholmes. \textbf{kevin} and \textbf{caitlin} kept me honest to
myself. thanks, kevin. thanks, caitlin.

ultimately, all praise go to \textbf{\tin{Amazon}} and all errors stay here
with me. thanks, \tin{Amazon}. we love you.
