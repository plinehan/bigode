\mychapter
{chapter_corn.jpg}
{beet soup . texas caviar}
{soup and salad:\\
chilled cream of beet/adzuki soup\\
with texas caviar}

\section{texas caviar}

i'd never heard of it before but apparently when matt came he made a
large salad out of black-eyed peas and available delectables and this
is what he called it ---

\begin{ingredients}
  \item a pressure cooker full of black-eyed peas\\
  \textrm{or}\\
  the black-eyed peas you didn't use in your veggie burgers
  \item onion
  \item garlic
  \item cilantro
  \item tomato
  \item lemon
  \item black pepper
  \item salt
\end{ingredients}

chop all the fresh and lovely spices together and mix with the
black-eyed peas. there should be enough green, white, and red to
create a sparkling taste sensation (other than the baseline beaniness)
and maintain visual stimulation. the lemon juice plays the sorcerer,
marinating the divergent elements together until they form a pungent
and unified whole.

one\\
people\\
under\\
Management

as the old sages used to say.

\subsection{easy \textit{PATTERN} to follow with bean salad}

so there is a Beauty and a Yoga to cooking too much so you have a
creative amount of leftover raw materials to work with for tomorrow's
lunch. i always make a full pcooker of beans even if i don't want so
many veggie burgers or tacos, just to have extra around for Elijah or
Experimentation. bean salads are satisfying in that the youthful vigor
of their acidity and variety of spry ingredients often belie a
powerful nutritional undercurrent. each bean is a unique and holy
individual and must be spiced accordingly, but i've found stability in
my desire for:

\begin{ingredients}
  \item chopped onions or green onions

  \item healthy dose of vinegar or lemon juice (the starch in the
  beans tends to get mushy and lean towards fermentation without a
  good Acid kick)

  \item leafy loving from a fresh herb like cilantro, parsley, basil,
  or mint
\end{ingredients}

other spices --- cumin, oregano, dill, soy sauce, ginger, garlic ---
vary from bean to bean: follow patterns and spice combinations from
cooked dishes (with what do you refry them?) for inspiration.

\subsection{a variation on texas caviar}

\subsubsection{garbanzo bean salad}

\begin{ingredients}
  \item rinsed cooked garbanzo beans
  \begin{ingredients}
    \item celery
    \item green bell peppers
    \item large slices of tomato
  \end{ingredients}
  \item a few scallions
  \item some feta cheese
  \begin{ingredients}
    \item fresh dill
    \item olive oil
    \item tahini
    \item lemon juice
    \item black pepper
  \end{ingredients}
  \item salt
\end{ingredients}

chop the celery and bell peppers together and mix with the tomato
wedges, chopped dill, and garbanzo beans. separately crumble the feta
and mix gently with diced scallions. make a dressing of olive oil, a
little tahini, lemon juice, and black pepper to pour over the bean
mixture. top with the feta and green onions.

\section{cold cream of beet adzuki soup}

i know it sounds strange but it's awesome and the best color you could
find. ever.

\begin{ingredients}
  \item a couple of beets
  \item a few cups of leftover adzuki beans
  \item soy sauce
  \item ginger
  \item tahini
  \item rice wine vinegar\\

  \item optional white wine
  \item optional coconut milk
\end{ingredients}

\begin{algorithm}
  \item chop your beets thinly and saut\'{e} them on medium heat with
  minced ginger. cook until tender, adding water or wine if necessary
  to prevent browning.

  \item when tender, blend together until uniform and continue
  blending with a few tablespoons of tahini 3. when you taste a nice
  balance between the beets and tahini, blend in the adzuki beans.

  \item continue blending, using the adzuki bean cooking water, white
  wine (one cup maximum), or coconut milk (no limit, soldiers) to thin
  to an appropriately soupy conclusion.

  \item mix in soy sauce and rice wine vinegar, a little at a time,
  until the soup has the right amount of salt and sour. the tahini
  should manifest as a subtle background creaminess.

  \item chill and serve garnished with mint leaves.
\end{algorithm}

\subsection{primordial soup \textit{pattern}}

it's chapter eight and if you're literate at all you must understand
that COOKING IS SO EASY. you don't even have to do anything --- some
remote ancestral God and an army of Mexican farm workers have
magically brought edible shapes to your kitchen and all you have to do
is eat them. cooking at its most difficult involves splashing hot
water and a dull knife. there's really nothing to it, and there's
nothing to prove there's nothing to it like soup. the general idea is
to cook whatever you have To Shit in hot water and salt. you can
pur\'{e}e the result if you feel like using the words ``consomm\'{e}''
or ``bisque'' but then you have to wash the robot. if you're feeling
particularly lazy and don't even cut the potatoes more than once
before dumping them in, label it ``hearty'' or ``old-world''.

i've often wrestled with myself in those pre-dawn post-enlightenment
hours where the ghost of lucidity eludes me amongst pickle jars of
cumin and urad dal --- should i even write this chapter? what's the
goddamn point if it's so self-evident. but, to quote robert hass, ---

\smallskip
\makebox[1cm]{}After a while I understood that,\\
\makebox[1cm]{}talking this way, everything dissolves: justice,\\
\makebox[1cm]{}pine, hair, woman, you and I
\smallskip

--- and everything we do in this beautiful human tragedy is so petty
and obvious anyhow (and therein lies its majesty) that i damn well
better write this chapter.

so, soup. we didn't make many soups in brasil because it was always so
hot and everyone was high on fruit all the time anyhow. when we did
they were either coconut milk stews (like moqueca), leftover bean
affairs (see below), or cold and refreshing (see above).

most standard vegetarian soups start off like an innocuous stirfry and
only continue to descend further into mediocrity. saut\'{e}ed onions
are met and cooked with peppers or perhaps carrots, water and other
sundries are added, and the whole affair simmered on the back burner
while the bread finishes baking and we, everconscious of our ant
brethren, clean the kitchen.

as is generally the case, everything gets more interesting with a hit
of cream, coconut milk, or pur\'{e}ed avocado at the end to bring a
watery compote to thick and satisfying broth. cheese and
bread-products (toast, croutons) also excel in this arena.

\subsection{variations on the soup}

\subsubsection{vegetable black bean soup}

pretend you are making a stirfry with onions, green peppers, and
carrots. add the onions first and saut\'{e} until translucent. spice
with garlic, cumin, oregano, and whatever else you've grown into doing
as a chef. add the peppers and carrots and cook vigorously for a few
moments.

at this point, a white elephant saunters into the room and reminds you
that NO SIR you are actually making a soup. down goes the heat
(medium-low) and in go the liquids --- pcooked black beans, a few
chopped tomatoes, some olive oil, salt, and extra water to boot. you
stir and reduce for the better part of a liter of beer (\gls{por
cozinheiro}), check for
thickness and life of the broth. if it tastes like water, add more
salt and fresh herbs and continue to reduce. if it tastes like salty
water, add more fresh herbs and some vegetable greens if you have them
--- or anything quick-cooking to flavor the broth.

please, for the One Love, do not use a processed cancer pill flavor
cube to liven up your broth. suffer through our communal ignorance
(remember dostoyevsky --- we are each guilty for all), slurp your
watery soup, and endeavor to do better next time. really. thank you.

a little cayenne pepper or crushed red pepper flakes makes the world
go around with this one. i also like to serve it with raw diced
onions, fresh ground toasted cumin, and grated cheese.

\subsubsection{lily's thai soup}

now lily is really beyond the scope of this book. i'll try to do some
justice to her soup, if possible.

\begin{algorithm}
  \item begin to stirfry whatever vegetables you have available.

  \item manifest the soupiness

  \begin{algorithm}
    \item get a local brother to harvest some wild lemon grass

    \item saut\'{e} (lots of) minced ginger and garlic and spicy hot
    chiles in oil
    
    \item when brown, add water, lemon grass, and coconut milk
  \end{algorithm}

  \item when the vegetables are almost done, let the two become one

  \item adjust saltiness with soy sauce or fish sauce (go ahead. live
  with yourself)

  \item when ready to serve add chopped cilantro and basil, simmer for
  a couple of minutes, and present with brio.
\end{algorithm}

\subsubsection{a simple gazpacho}

gazpacho is ancient spanish for salad + liquid.

cut together, preserving the juices, of
\begin{ingredients}
  \item tomato
  \item green, red, bell, anaheim peppers (small pieces)
  \item cucumber (diced)
  \item onion (quarter moons)
\end{ingredients}

salt the mixture to draw out the juices while you make a power pack of
\begin{ingredients}	
  \item olive oil
  \item vinegar
  \item minced garlic
  \item lots of black pepper
\end{ingredients}

tomatoes have a plurality, and it goes on down from there. both the
oil and vinegar add strong and important elements --- the vinegar
particularly to marinate the soup together as it chills (at least a
couple of hours). i like to add a cup of wine or vermouth as well,
barring ramadan, to really get the party started.
		
serve with healthy amounts of chopped parsley or cilantro according to
your clientele's neuroses and food allergies.

\subsubsection{cold cream of carrot consomm\'{e}}

saut\'{e} your carrots, onion and ginger in butter. as the carrots
finish cooking and gain a tender edge, mix in some brown sugar for a
light glaze. cook a few minutes more and transfer to the robot. blend
with a little soy sauce, vegetable stock, and whatever cream (dairy or
coconut) floats your roots.

if by any stroke of cosmic luck you have fennel available, use it. you
can use the chopped bulb instead of onions in the stir frying scene
and a few sections of the leafy herb in with the soupy mass. taste the
soup and water/coconut down to taste.

serve with gomasio (toasted sesame seeds and salt) in intricate and
meaningful mandalas.
