\mychapter
{chapter_corn.jpg}
{stirfried vegetables . coconut polenta . cabbage slaw}
{peanut-dressed stirfried vegetables lounge atop coconut polenta as
soy-ginger cabbage slaws pleasant amusement}

3.1  coconut polenta

1 packet or a few cups of polenta
boiling water (more than 2:1)
salt
coconut: fresh, dry, or milk
a lemongrass stalk or two
some black pepper

boil the water. get a whisk or other implement with which you can stir with the steady vigorous focus of a true culinary Jedi. if you can find lemongrass, you can add it to the boiling water. the tea/soup this makes is excellent and vital for your continued survival on this disintegrating planet --- MAKE SURE TO DRINK AT LEAST A LITTLE BIT. when the water is hot, turn down the heat to medium-low, remove some of it (\onethird perhaps) for ``just in case you need more''.

add your coconut product and mix before adding the polenta. if the coconut is dry (desiccated), allow it to rehydrate and soften for a couple of minutes before adding the corn. remember: polenta stands united and will clump together in culinary disobedience unless you (The Man) whisk down with surety and force. pour the polenta in a slow, steady stream, constantly stirring. when you achieve a smooth consistency, evaluate.

if it's looser/soupier than you like your porridge, lower the heat and continue to stir occasionally as you finish the rest of your preparations. in a few minutes it should thicken up nicely. if it's tighter than you like, add some of the reserved water or more coconut milk until you get Just The Right Thing.

do not take the polenta off the (low) heat until you are ready to serve --- the texture and flow change completely as it cools. make it last and when your fools and lovers are ready to eat, stir in the freshly cracked black pepper and plate immediately.


3.1.2 the pattern amongst polenta's grains

polenta is (perfectly and paradoxically) both simple in its cooking and varied in its possible uses. the first incarnation of polenta is a sort of thick corn gruel obtained by stirring together ground corn with boiling water. this porridge can be cooked to a variety of consistencies and adapts to its form upon cooling, so it can easily be used for odd shapes or innovative presentation ideas.

for its initial cooking (described above) most people prepare polenta in one of three ways
	-	a loose gruelly porridge, with vegetables or spices mixed in or served on top
	-	a firmer more gelatinous porridge, served with an ice cream scooper and holding an attractive shape on the plate
	-	a focaccia-style creation, poured until flat and baked until firm

in the initial recipe, being a brasilian restaurant, i cooked the polenta with coconut and lemongrass. if you're not in brasil you probably won't (and shouldn't) follow suit because the coconuts, after such long and bumpy travels, will not be in a happy nutritious mood. other additions could be:
	-	chopped chipotle peppers and lime juice (the lime at the end)
	-	fresh mediterranean herbs (lavender, rosemary, sage, oregano, marjoram)
	-	chopped greens and garlic: beet greens, collard greens, mustard greens, chard
	-	milk instead of coconut milk, apples boiling in the water instead of lemongrass, and cheddar cheese with black pepper at the end

to achieve the second from the first merely continue cooking. as more water evaporates the porridge will become jello-like and hold together better. keep the polenta at a warm temperature (warm oven, warm water bath) until serving.

to achieve the third you can passively pour the hot polenta into a tray and let it cool, or bake the water out of the polenta in its new shape. i've found it easier simply to let it cool to room temperature and then even put the entire tray in the fridge for a couple of hours. make sure the tray has been lightly greased so it's easy (flip it over) to get the polenta out. 

when you want to serve the formed polenta:
	-	remove it from the fridge, allowing it to return to room temperature
	-	cut it into whatever clever shape you desire
	-	sear it in oil or butter on the stove/grill, forming a tasty crust on each side.

3.1.3  variations on coconut polenta

lily's polenta breakfast

so there's this girl lily who lived with us for months and months and it's this whole other story but she dug the polenta and sometimes would have it for breakfast with nothing but some sugar or honey. however, if you live in the first world or are trying to impress somebody's parents you can put all kinds of breakfasty additions in your exotic corn porridge.

start by cooking the polenta (as above) and maybe using milk (if you're into that kind of thing) or coconut milk with your water. add sugar or maple syrup with the polenta and stir towards conformity. when it's ready open your pantry and inject whatever you like:

	raisins
		chopped dried fruit
						walnuts
	grapenuts
			toasted coconut
					picante
		yogurt
			granola	

polenta with leftover saut\'{e}

this chapter's title dish takes freshly stirfried vegetables and mounds them atop a warm polenta porridge for a hearty, satisfying meal.

the next day you're going to have soft wilted vegetables and cold hard polenta sitting outside amidst empty beer bottles and scraps of purple ribbon on your floor. maybe some seashells and hemp bracelets in the corner and your roommates nowhere to be found. you're going to need some lunch before you can really focus on Ending Poverty or whatever else it is they happen to be selling on the culture channel.

heat up a cup of water on the stove; add leftover polenta when hot, mix to the best of your ability.
when more or less conformed, mix in a few chopped handfuls of last night's vegetables.
while they heat together, dice a few green onions and juice a lime or lemon.
take the pot off the heat, stir in some more black pepper, the scallions, and citric juice.

the fresh touches at the end (cilantro would also be nice) help one rediscover the vitality of the food. they are essential. 	

seared sun-dried tomato polenta

a more traditional polenta variation from our brothers on the mediterranean

polenta
water or stock
sun-dried tomatoes
white wine
onions, diced
garlic, minced
basil, chopped
cheese, grated

if your tomatoes are already wet and dripping with olive oil, chop them and you're ready for business. if they're still dry soak them for a few hours in white wine (if that's legal and desirable for you) or water. if you don't have a few hours, rehydrate them in hot water or wine for 15 minutes. when they're soft and wet, chop them and set aside.

heat a pot for the polenta with a little olive oil in the bottom. saut\'{e} your onions and garlic together on medium heat until they go from white to clear and feel soft and easy.
when the onions are translucent add your stock if you have some or salted water if you don't. when the water starts to boil, reduce the heat and add the polenta. if you forget to turn down the heat the polenta will boil over and you will have a hilarious mess. as the polenta cooks and smoothens, add the basil, cheese (parmigiano or romano if you're in the first world), and black pepper.

when The All is One pour into a tray or baking dish to cool (as directed above). let it relax to room temperature, chill in the fridge, and relax again to room temperature. while it's chilling you can press whatever you want into the top of it for fossilized spice or herb bits in your polenta. i use thin slices of garlic, whole rosemary branches, and tomato rounds.

when you're ready to serve, cut the polenta into pieces (squares, triangles, circles using a coffee cup, narrow rectangles to be seared on their sides) and perhaps in half heightwise if they're thick enough. heat some oil or butter in pan (good high heat) and reverently toss a peeled clove of garlic into the oil. when the oil is hot and garlic is fragrant, begin searing. each piece should cook for a minute or two on each side: enough to pop, brown, blister, and warm all the way through. it helps to cover the pan as you're searing the second side to add steaming action and make sure the middle is hot --- especially for thick pieces. take care with slices who have cheese or fresh tomato on the outside edge --- they will be blister faster, making them more and more attractive until they burn and are ruined (FOREVER).

cover attractively with sauce or saut\'{e}ed vegetables and serve.

3.2  stirfry vegetables

nothing hits the Spot and the Complementary Contrast better for me than sizzling crunchy vegetables over a soft warm grain. you work for half the meal and are comforted by the other. take:

onions
green onions
whatever vegetables they have at the market (carrot, cabbage, peppers, broccoli, cauliflower, beets, cucumbers, green tomatoes)
garlic
ginger

wash all your vegetables and peel the pestilential ones. cut them into thin longish slices that accentuate their shape and color. be adventuresome and don't worry about the forks of your fellow men and women. they will find a way.

heat some cooking oil (peanut oil, canola oil) in a pan, on high heat, and add the onions when hot. stir them well, constantly in motion: we want them soft and translucent, neither brown nor burnt.
	while the onions cook, chop together a few cloves of garlic and pat's thumb of ginger. add the ginger and garlic when the onions are soft.
cook briefly and start adding the vegetables. hard items like carrots and beets should go in first, then cabbages, peppers, etc. spinach, greens, pineapple or other fleshies can go in last. the key is to cook everything exactly to the right texture so nothing is raw and nothing is mushy.

when everything is almost done, turn down the heat and add your sauce: either the peanut sauce (below) or any variation you see fit. mix together on low heat for a minute or two, until everything is cooked perfectly and has the right consistency, and serve immediately.

3.2.2  stirfrying vegetables following the pattern

if you're not going to eat your vegetables raw and really maximize the work they've done for you (and the work you have to do for them), then it's probably best to have them cooked as little as possible. stirfrying cooks vegetables at a high heat for a short amount of time, allowing sweetness to develop before the soggy sets in. it's work intensive in the sense that it requires a lot of chopping and a lot of attention, but these obligations are better seen as trainings --- different yogas or meditations to help you relax into your true, eternal, selfhood.

1. wash, peel, and chop all your vegetables. take up a lot of room if you can.
	- it's easier and safer to chop vegetables with a flat side down. many vegetables are not born with a flat side and it is up to you to give them one. luckily, with a knife, this is not hard.
	- body- mind- and spirit- altering chemicals (nutrients, pesticides) congregate in the peel. you may want to remove it.
	- chop slowly and with dedication. relax. let it be a medication. having a bounty of vegetables around you is in itself a gift. having the opportunity to feed your fellow human is in itself a gift. if you focus on these presents you will grow full and fat before you have heated a single drop of oil.

2. appreciate them. the color, the size, the texture.
	- there are millions of US who do not eat when we want to.
	- there are millions of US who only eat the insides of cans and boxes. artificial colors and illusionary sizes.
	- you are truly blessed to be looking at a cutting board or countertop full of vegetables.

3. highly heat a small amount of oil in a large amount of pan.
	- use peanut, canola, mustard, or whatever the locals do.
	- olive oil and sesame oil will burn immediately, so if that's all you have skip down to dry-frying (below).

4. fry your onions first.
	- until translucent. onions take more stages of cooking to get to the taste and texture we want. so it's good to privilege them.

5. fry your garlic and ginger next.
	- just for a minute until they begin to release they're goodness. you don't have to use them either. you don't have to do anything you don't want to, regardless of what the police or your middle school teacher tried to tell you.

6. fry everything else in order of size and hardness.
	- this is the zen or dance of timing. it's very important and will take you a few tries. it's worth it. there is an alternative (below) if you really think this desperate mix of on the fly density vs. surface area calculations is not for you.
	- good vegetables include:
	carrots
	xu-xu	(see chapter five)
	bell peppers of various colors
	cabbage
	cauliflower
	broccoli
	lotus root	(if you're really living in style)
	bok choi	(or anything foreign that sounds like it's from a stirfry culture)

	mushrooms

7. add your sauce.
	- some sauces (that have sugar or cornstarch or lots of liquid) like to be cooked to thicken and gloss. others want to stew to exchange flavors. and others just need to be mixed. generally the heat of the stirfry will be too hot for the sauce, so you can either remove the pan from the heat or turn it down (or off).

that's just the standard way i've learned of cooking together different vegetables. there are a number of other easy techniques to get totally different flavors and textures out of the same vegetables:

dry-frying:
	oil. if you're stranded on some lakeside in patagonia and don't have any or are prison to your ideological beliefs and can't find any organic non-gmo varieties at the local hypermarket, dry-frying is for you. it works best for a fajita-style mix of onions and peppers, but i've used it with great success on any vegetable that carries enough water to cook itself. as i understand it, the chemical premise is that the high heat steams out the internal water of your vegetables, cooking them from the inside-out as the hot metal sears them from the outside in. you should end up with some blistering and charring, to a degree you can control by how often and how vigorously you stir. 

	heat a pan to very hot. cheap aluminum or steel pans work the worst and will scar and burn easily. heavy steel pans, cast-iron and of course magical poisonous teflon work the best. throw in your vegetables all at once --- onions and peppers or carrots or peppers and carrots or broccoli or carrots and beets or whatever. let them sit and cook at rest for a moment, and then begin to stir them. relax and let your intuition guide you --- you will be drawn to stir or agitate that vegetable which is about to blacken until you are dancing with the heat of the pan, attuned to the microvariations of temperature, balancing the steaming and charring action. when the vegetables are appropriately soft, add roasted cumin, salt, or whatever other spices you desire. stir a few moments longer, and remove from the hot pan.

more authentically chinese stirfrying:
	i've never done this but i feel i have to include it because it might be true. apparently stirfrys are traditionally made by frying each vegetable separately, then combining and heating the different elements together in the end. this technique is way beyond my levels of dedication and patience and i could only suggest it as a philosophical exercise, or for those who absolutely can't handle the idea of adding each vegetable to the mix at Just The Right Time to get the cooking to come out even.

gentler proven\c{c}al-style saut\'{e}s:
	many cultures prefer to cook their vegetables on lower heat for a slower softer cooking. natural sweetness develops better, there is less chance of burning, and less work and attention required of the cook. follow the normal directions but keep the heat on medium, don't pay careful attention, and feel free to cover to encourage steaming (they're going to be soft anyhow).

3.2.3 varying stirfried vegetables

fajita style onions and peppers	

follow the dry-fry technique from the beginning, using onion rings or half-rings and long strips of bell peppers. chile peppers work well too for a much more intense experiences. as they soften and start to blister i'll throw in a couple of chopped cloves of garlic to fry for a few minutes, as well as ground roasted cumin and a little oil (if i have some around). this cuts the burning and everything mixes together softly and easily. these vegetables are great in tacos, as a sidedish, appetizer, over polenta, or anywhere else.

instant sauerkraut

traditional sauerkraut is a conspiracy between various species of airborne bacteria and the human consumer, to preserve vegetables for months (or years) with added flavor and nutrition. this version replaces centuries of careful knowledge and technique with vinegar, which has a similar taste. using a flavored vinegar (see chapter two) would of course be preferable to the clinical distilled white you'd take home from The Store.

as much cabbage as you can eat or chop
some spices: dill seeds, caraway, black pepper, minced garlic, salt
a sliced apple or two
vinegar

use the standard technique for saut\'{e}ing vegetables --- medium heat in a little bit of oil, stirring enough so it doesn't stick or burn but not obsessing because the heat isn't quite dangerous enough. when all the water has been exorcised from the cabbage, add your spices and apples and cook together with the leaves for a couple of minutes. when the soft feels in danger of drying out, add enough vinegar to truck the cooking along. as the veggies steam more and the vinegar reduces, taste for acidity. it should taste sour and strong but not so much that you can't eat it. add more vinegar and cook until it's as strong as you like.

cover while people find the right chairs and wash their right hands.

onion, zucchini, and carrot taco filling

diced onion
half moon zucchini
cubed or grated carrot
cumin
minced garlic

saut\'{e} these vegetables on medium-high heat, following the technique above which places the onions first and adds the zucchini and carrots when the former are soft. i add garlic and cumin to the onions a few minutes before the other vegetables, and squeeze on lemon juice and salt at the end. the zucchini have a higher water content and will finish softer than the carrots --- the more you stir them, the saucier they will become.

saut\'{e}ed potatoes

one of my favorite comforting weekend breakfast recipes, this saut\'{e} accepts a wide variety of seasoning combinations and can fill many roles on the p(a)late. the key is in the small and even sizing of the potatoes

some potatoes cut into small frenchie cubes (brunoise)
oil
salt
pepper

to make a brunoise cut, cut off a long side of the potato to give a flat surface. roll the potato onto this flatness (for security) and make a similar cut on the next side (for posterity). continue slicing down the potato, holding it together with your hand above the knife if it starts to fall apart. when you've sliced the potato into an array of thin (pinky thick) slices, turn the entire stack on its side (the second flat surface you created) and repeat the process. you will now have a matrix of matchsticks.

apply a ninety degree transform to your matrix and chop into cubes. the important part is to have a flat surface down at all times to allow yourself the peace of mind to make firm swift cuts without worrying about slippage and medical bills and your lack of insurance coverage.

heat more than enough oil to cover the bottom of the pan on medium heat. when loose and flowing, add the potatoes. they are full of starch (glue) and will want to stick to the pan, so it is your sacred duty to stir them often enough to prevent stickage. keep the heat on medium, cover the pan when not stirring, and cook for 10-15 minutes. they will eventually brown on all sides (if you're stirring evenly) and be cooked through (the smaller the faster). add salt and pepper, toss and stir a minute longer, and serve.

if you run into serious sticking problems, assuage the situation by adding a little more oil to the pan (not directly to a potato --- he will greedily absorb it all leaving nothing for his brethren). adding water to the pan, which may at some points seem like a good idea, will likely cause over-steaming and a mushy potato result which you will have to rename before serving.

3.3 soy-ginger cabbage salad

one of the most popular dishes (among gringos) at O Bigode, this salad marinates thinly prepared vegetables with a powerful dressing.

healthy cabbage
bunch of carrots
pair of beets
green and white onion
sesame seeds

	shred cabbage
	grate carrots and beets
	thinly slice white onion
	dice the green ones,
	and toast the sesame seeds.
	
	dressing:
tamari
tahini
rice wine vinegar
minced ginger
minced garlic
honey
sesame oil

	combine and mix vigorously. start with \onethird cup of tamari and \onequarter cup of tahini; taste after each ingredient.

3.3.2 the pattern of this cabbage salad

three elements are key to this type of salad:

-	vegetables cut with high Surface Area To Volume Ratio (SATVR)
-	some time for the dressing to marinate and impregnate the high SATVR-cut vegetables (SATVRCV)
-	a good dressing

within these constraints you can thinly slice (or grate; much easier with a grating robot) any vegetables together, let them soak with some good flavors, and enjoy.

3.3.3  variations on soy-ginger cabbage salad

a simpler teriyaki-style sauce

if the soy-tahini-ginger is a little much for you, this provides a simpler dressing or sauce for a vegetable stirfry

	orange juice
	soy sauce
	honey
	a little bit of minced ginger

mix the ingredients together until slightly viscous. if using for a stirfry sauce, be sure to cook a little bit on medium heat to thicken, and then toss in sesame seeds during the last minute or two of heat to speckle your vegetables.

asian dressed carrot apple salad

carrots
cucumbers
apples

grate your carrots. if you are cooking for people who like vegetables and like asians, use 1 carrot per person. otherwise use \onehalf. always think about future versions of the self (lunch tomorrow) and any guests who might show up (doid\~{a}o or elijah or otherwise).

mix the grated carrots with a liberal amount of rice wine vinegar. let them sit and negotiate while you finish grating.

peel and de-seed the cucumbers (using a spoon works dandy, just scrape down the line) and grate. use half the mass of cucumber as of carrots. squeeze them to remove the water (most of it) and reserve for a nice vodka cocktail. 

peel and grate an apple or two as well.

drain the vinegar if you added too much, combine all the ingredients, and add the sauce:

soy sauce
toasted ground sesame seeds
minced ginger

in general, the more sesame seeds the better, unless somebody is deathly allergic to sesame (ask).

apples in bahia?

every once in a while we would get apples. most crops grow year round in bahia but apples were cultivated in the south and harvested in the southern fall. so by october these apples had been in cold storage for six months and shipped from at least a 24 hour busride away. they were mediocre at best and cost over three times as much as mangoes.

clearly this dish would be hard to do without the carrots. everything else is negotiable. the combination i like is the sweetness of the carrots soaked in the sweet/tart of the vinegar. it's best with the rice wine (and indeed --- just the carrots and vinegar is common in vietnamese food) but i've made it with just apple-cider vinegar as well. i never saw a pear in brasil but little cubes of firm/crisp pear instead of apple would be perfect. you wouldn't even need the soy sauce.

3.4  peanut sauce

making a decent peanut sauce is incredibly easy, and making a great one takes about three tries. the key is to balance the creamyfatty taste of the peanuts (which we in amerika are very attracted to) with the flavorings that lend subtlety to the dish.

water
peanut butter
soy sauce
rice wine vinegar
honey
garlic
ginger
something hot

water your peanut butter down by whisking them together on low heat. take the sauce off of the heat and start added flavorings in the order listed. when it's salty enough, stop with the soy; when it's sour enough stop with the vinegar. etc. trust yourself: if you like it so will everybody else.

3.4.2  peanut sauce patterns

i couldn't find anywhere else to put high-fat vegetable variations so here's where they're going to be. coconut, peanut, and avocado -based sauces are similar in that they start from a creamy fatty base and flavor it with the strong additions necessary to compete with so much luscious vegetable richness.

if you have access to peanuts or coconuts, THERE IS NO CONCIEVABLE NEED to buy peanut butter or coconut milk.

how to make peanut butter

take your roasted peanuts and put them in your robot. plug in and turn on.
through the curtain of schrappy noise you will notice the peanuts go from whole to chopped to powdered to a big oily ball to a smooth creamy butter. after the ball relaxes into a butter, blend a little longer to force even more oil out of the nut. taste and add salt if you want ``salted peanut butter''. thin with water if you're making a peanut sauce. there is absolutely no need to add oil.

how to make coconut milk

if using desiccated shredded coconut, soak in water for an hour and treat as fresh coconut.
if using fresh coconut, blend your coconut chunks in enough water to smoothen. this will take few minutes.

the smooth paste you get is coconut milk. what you find in the inside of an unripe coconut is coconut water. they are two different things. in the evolution of the holy coconut, actually, it is the coconut water that solidifies into the coconut meat, which dries into the flesh of the coconut fruit. to wit:

coconut water $\rightarrow$ wet meat $\rightarrow$ dry meat $\rightarrow$ grated coconut $\rightarrow$ coconut milk

3.4.3  variations on peanut sauce

coconut curry

a simple thai coconut sauce could be saut\'{e}ing the following (as above) and letting the result simmer in coconut milk while you prepare everything else:

	onions
	garlic
	green chile
	ginger
	lemon grass

basil and cilantro in large quantities help turn the coconut milk green. this sauce then gets added to whatever vegetables you've just cooked, either merely to coat or as a sort of stew.

coconut chutney w/ mustard fenugreek

you can get a lot of insight into a culture from its markets and thriftstores. one device you could find in every kitchen/restaurant supply store in salvador looked like a horizontal orange juicer mounted on a hand crank (or electric), cost 40 points, and was used to grind out the meat of a coconut. so all you have to do is split it with your fac\~{a}o, drain the agua, mount it on the provided bracket, and crank until all the delicious meat piled up in front of you. definitely the way to go if you were making cocada or a lot of this chutney (say for a wedding) and were intent on using fresh coconut.

luckily, the recipe adapts well (as in, hell, everything adapts well if we adapt well, it doesn't have enough ego to care) to desiccated coconut: just transmute it into coconut milk (see above).

coconut milk (either newly made or purchased) is the base of the chutney. the other two components are fresh spices and roasted spices. for the fresh spices i use a mixture of minced ginger and green chiles, with more of the latter than the former. as for the dry spices:

heat a small amount of oil in a pan. when hot add 1 tbsp each of:
	black mustard seeds
	fenugreek seeds
	chana dal

they might be hard to find but here the flavor matters. dal just means split so i imagine you could break a couple of dried chickpeas and use that if you don't have chana dal in your house. chana dal is a really great lentil, however, and i recommend having it with you AT ALL TIMES. borders crossings and everything.

\footnote{fac\~{a}o (br) : large knife, machete}
\footnote{cocada (br) : coconut custard style brasilian desert}

let the seeds pop and roast in the oil and take off the heat when browned. grind in a mortar and pestle until powdered. there will be some fragmentary chunks and that's fine. mix in the oil and spices to the milk, ginger, and chile. the chana flour you've made should act as a thickening agent, to help the sauce hold together. if it doesn't work, it's because you made the coconut milk too loose or didn't toast/grind enough chana dal.

it doesn't have to be very tight anyhow for dipping samosas or empanadas, but works great as a sandwich spread or on pita if it's a little thicker (like mayonnaise).

sweet and spicy aguacates

guacamole, while famous in mexico and everywhere else, is basically unheard of in brasil. they have many types of avocados, delicious and smooth, and never put them near garlic. instead they like their avocados sweet, with sugar or honey on bread, or in smoothies. so to be fair i'll put them both, though we made the guac a lot more often than the jam

	guacamole:
		mix together roughly with a fork
		the lemon salt chile combination is really what's going to send you out there, and the rest of it is an important fashion accessory.

a few soft, ripe, dark avocados
half an onion diced small
two minced cloves of garlic
a little cayenne pepper
juice of one lemon, maybe two
a ripe tomato
fresh ground toasted cumin
salt
	

	doce de abacate:
		mix together until creamy
		as sweet as you can take it (the brasilians take it pretty sweet) and the texture to spread on bread. add ice water for an interesting smoothie.

a few soft, ripe, dark avocados
a few tablespoons of sugar or honey
a little bit of cream or coconut milk

\footnote{aguacate (sp) : avocado;   doce de abacate (br) : avocado jam}
