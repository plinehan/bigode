\mychapter
{chapter_carrots.jpg}
{some brasilian favorites}
{some\\brasilian\\favorites}

\section{for those of you who...}

\begin{enumerate}
  \item[a)] no longer believe in coincidence
  \item[b)] thought this might actually have some brasilian recipes
\end{enumerate}

i've tried to collect some of the more popular bahian recipes for the
Bigode community in a desperate quest for authenticity in a globalized
culture. perhaps, through a few queijos quente tropicais, we'll all be
brasilian one day: smiling, sexy, hungry, and 100\% NEGRO.

\subsubsection{cheese-egg}

butter both halves of an old glutinous white sandwich bun or french
roll grill or toast. do not clean grill beforehand.\\
while toasting, oil nearby grill real estate and fry an egg.\\
as the egg solidifies, lay a slab of yellowish cheese atop it.\\
the cheese melts as the egg finishes.

impurists have the option of searching out a few slices of underripe
tomato and yesterday's lettuce.

bread $\rightarrow$ lettuce $\rightarrow$ tomato $\rightarrow$ egg
$\rightarrow$ cheese $\rightarrow$ bread

\subsubsection{queijo quente tropical}

the brasilian version of yesterday's grilled-cheese sandwich, which
amanda discovered on the menu of one of our staple internet cafes. it
had always been there i guess, as old as pineapples and slavery,
cleverly hidden amongst wicker chairs, 500 mL bowls of
a\c{c}a\'{i}\footnote{a\c{c}a\'{i} com banana (br) : the amazonian
superfood} com banana, and the most amazing book i've been blessed by
browsing: \underline{frutas do brasil}\footnote{frutas do brasil (br)
: fruits of brasil, second only to \underline{the revolution of
everyday life} and \underline{the one-straw revolution}}.

use three thin pieces of white (or white/wheat) bread,\\
grill cheese slices between slices one and two,\\
pineapple slices between slices two and three.\\
assemble with lettuce and tomato\\

\subsubsection{acaraj\'{e}}

the premier bahian snack food, a large deep-fried falafel-esque
creature made from black-eyed peas that originated as a sacrament for
candombl\'{e} rituals.

soak ``feij\~{a}o fradinho quebrado'' (broken black-eye peas) over
night. if you can't find the broken ones, briefly blend whole ones
until they break.  rinse and drain the beans, soak again, and rinse
again. continue every couple of hours until you've rinsed away all the
hulls, until you see just the whiteness of the beans.

robot the soaked, rinsed, drained, hulled beans with an onion. add
small amounts of water to facilitate a smooth blending. you will end
up with a foamy pancakey dough. let the dough rise for a few hours (in
brasil) or a day in a preheated oven (in wintry temperate climes). it
will ferment and conspire with yeast in the air to bulk up and gain a
sour edge.

stir the dough, form with your hand into firm patties the diameter of
cheap hamburgers and much thicker. deep fry in hot oil --- the baianos
use the wonderfully fragrant and fattening azeite-de-dend\^{e}. it's
red and cooks the acaraj\'{e} a deep reddish brown.

to serve, hot!, the brasilians slice down the center and add some of the following:

\begin{ingredients}
  \item vatap\'{a}: coconut shrimp paste
  \item caruru: slimy curried okra
  \item pimenta: spicy chile sauce
  \item salada: unripe tomatoes and onions
\end{ingredients}

i never got close enough to a proper baiana to learn how to make
them. they're sacred anyhow and probably best left to the priestesses
and professionals.

\subsubsection{peixe frito}

every restaurant in brasil had one thing, fish, cooked in two ways:
frito and moqueca. it always came with rice and beans and salad
(tomatoes and onions) and hot sauce and of course lots of beer. and it
was truly excellent. all of the restaurants were on the beach and the
women would cook what the men had brought in from the sea that morning
--- usually peixe vermelho.

clearly this isn't a vegetarian recipe, but it's not much of a recipe
anyhow, so i'm not sweating it either.

\begin{algorithm}
  \item go outside and catch a red fish (peixe vermelho\footnote{peixe
  vermelho (br) : red fish})

  \item clean it with a sharp tetanic looking knife and when the
  customers order

  \item marinate it for 15 minutes in garlic, onion, salt, and lemon
  juice.

  \item after they've had a few rounds of beer, gently place the
  entire fish in a vat of hot oil and fry it until done

  \item the flesh will move away from the bone as the skin
  crispens. it's done.

  \item serve on a bed of lettuce with steaming bowls of pinto beans
  and rice, and another round of beer.
\end{algorithm}

\subsubsection{feij\~{a}o tropeiro}

feij\~{a}o tropeiro, as i understand it, is a dish from the south of
brasil (perhaps minas gerais) where the familiar black-eyed pea is
saut\'{e}ed with spices and dusted heavily with the ubiquitous farinha
de mandioca, so it gets a dry, grainy texture. it's good if you have
plenty of beer on hand to wash it down. luckily, brasilians always do.

saut\'{e} diced onion, diced tomato, a little garlic all together in a
little cooking oil. when the onions have mostly cooked and the
tomatoes are pretty dry, add the black-eye peas and cook
together. when the onions are fully done the flavors will have unified
(UNITY!) and you can squeeze in some lime juice and dump in a good
amount of farinha. keep stirring and adding farinha until the beans
are well-bespeckled and you get that dry, chalky feeling just looking
at it.

salt and serve with coentro picado.

\subsubsection{moqueca}

moqueca is bahia's most famous dish, a fish stew simmered in
azeite-de-dend\^{e}, leite de coco, e tomate. i documented how to make
it before even making it to bahia, from a kind restaurateur in paraty
(near rio de janeiro).

fernando says, serve de 2 a 3 pessoas:

\begin{algorithm}
  \item heat a clay pot with tall fire for at least ten minutes

  \item locate 50 mL of azeite-de-dend\^{e}, 50 mL of cooking oil in
  the pot

  \item toss in a full hand of

  \begin{ingredients}
    \item chopped onions
    \item chopped peppers
    \item chopped tomatoes
  \end{ingredients}

  \item add a little salt and parsley

  \item add the filling when the mixture starts to brown: fish,
  shrimp, octopus, what have you ... \\if using a frozen fish product,
  it should have been defrosted by now

  \item then 180g de polpa de tomate\footnote{polpa de tomate (br) :
  tomato pulp or paste}

  \item then 180g de leite de coco
\end{algorithm}

cook for 8 to 10 minutes. when well bubbling, make the pir\~{a}o
(gravy) by taking some of the liquid from the pot and mixing elsewhere
with farinha de mandioca and a little leite de coco.

ending note: when i watched his cooks actually making it, they used
closer to 100g of polpa de tomate and 300g of leite de coco.

\bigskip

naturally.
